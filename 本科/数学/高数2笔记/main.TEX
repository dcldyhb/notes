\documentclass[lang = zh , final , oneside , openany , titlepage , zihao = -4 , linespread = 1.3 , baselineskip = false , cjk-font = windows , text-font = newtx , math-font = newtx]{sjtureport}

\usepackage{amsmath}
\usepackage{amsthm}
\usepackage[hidelinks]{hyperref}%目录超链接并隐藏
\usepackage{graphicx}%图片路径支持
\usepackage{gbt7714}
\bibliographystyle{gbt7714-numerical}
\usepackage{booktabs} % 表格支持
\usepackage{longtable} % 表格支持
%\usepackage{subcaption}%小标题支持

%\sjtusetup
%{
%  info =
%  {
%    zh/title = {上海交通大学学位论文模板示例文档},
%    en/title = {A Sample Document for SJTU Thesis Template},
%    zh/author = {某某},
%    en/author = {Mo Mo},
%  },
%
%  style =
%  {
%    float-num-sep = {-},
%  },
%
%  name =
%  {
%    achv = {攻读学位期间完成的论文},
%  },
%}

\title{高数2笔记}
\author{dcldyhb}
\subject{高等数学2}
\keywords{上海交大, 饮水思源, 爱国荣校}

\begin{document}

\maketitle

\setcounter{page}{1}  % 将页码设置为1
\pagestyle{plain}     % 设置为普通页码样式
\tableofcontents

%\begin{abstract}
%本模板是上海交通大学本科生课程论文、学位论文、学术报告等文档的LaTeX模板,旨在帮助学生快速上手LaTeX排版。
%\end{abstract}

\newpage
\setcounter{page}{1}  % 将页码设置为1
\pagestyle{plain}     % 设置为普通页码样式
\chapter{重积分}
\section{重积分的概念和性质}
\begin{definition}
    设 \(D\) 是平面上的有界闭区域,\(f(x,y)\) 为 \(D\) 上的有界函数,\(I\) 为实数.若对 \(D\) 的任意分割 \(\Delta D_1 , \Delta D_2 , \cdots , \Delta D_n\) ,任 \((\xi_i , \eta_i)\in\Delta D_i(i=1,\ldots,n)\),作和 \(\displaystyle\sum_{i=1}^nf(\xi_i,\eta_i)\Delta \sigma_i\)(\(\Delta \sigma_i\) 为 \(D_i\) 的面积),总有

    \[\lim_{\lambda\to 0}\sum_{i=1}^nf(\xi_i,\eta_i)\Delta \sigma_i = I\]

    其中 \(\displaystyle\lambda = \max_{1\leq i\leq d} \{d_i\}\),\(d_i\) 是小区域 \(\Delta D_i\) 的直径,则称函数 \(f(x,y)\) 在 \(D\) 上\textbf{可积},记为 \(f\in R(D)\);极限值 \(I\) 称为 \(f(x,y)\) 在\(D\) 上的\textbf{二重积分},记作

    \[\iint_D f(x,y)\,\mathrm{d}\sigma.\]
\end{definition}

\begin{itemize}
\item
  \(\displaystyle \iint\) 积分号
\item
  \(D\) 积分区域
\item
  \(f(x,y)\) 被积函数
\item
  \(\mathrm{d}\sigma\) 面积元素(微元)
\item
  二重积分的几何意义

  \begin{itemize}
  \item
    当被积函数大于 \(0\) 时,二重积分是柱体体积
  \item
    当被积函数小于 \(0\) 时,二重积分是柱体体积的负值
  \item
    一般的,为曲顶柱体体积的代数和
  \end{itemize}
\item
  可积的充分条件

  \begin{itemize}
  \item
    \textbf{定理:} 若函数 \(f(x,y)\) 在区域 \(D\) 上连续,则
    \(f(x,y)\in D\).
  \end{itemize}
\item
  \(f(x,y)\) 在 \(D\) 上的可积性及积分值与其在 \(D\)
  内\textbf{有限条光滑曲线}上的定义无关
\end{itemize}

\section{二重积分的性质}

\begin{enumerate}
\def\labelenumi{\arabic{enumi}.}
\item
  \(\displaystyle\iint_D\,\mathrm{d}\sigma = \iint_D 1\,\mathrm{d}\sigma = A_D \quad\left(\text{D 的面积}\right)\).
\item
  \textbf{线性性:} 设
  \(f,g\in R(D)\),\(\alpha , \beta\),是任意常数,则
  \(\alpha f +\beta g \in R(D)\),且

  \[\iint_D (\alpha f +\beta g)\,\mathrm{d}\sigma = \alpha \iint_D f\,\mathrm{d}\sigma + \beta \iint_D g\,\mathrm{d}\sigma\]
\item
  \textbf{区域可加性:} 若 \(f\in R(D)\) 且积分区域 \(D\)
  分为内部不相交的子区域 \(D_1 , D_2\),则

  \[\iint_D f(x,y) \,\mathrm{d}\sigma = \iint_{D_1} f(x,y)\,\mathrm{d}\sigma + \iint_{D_2} f(x,y)\,\mathrm{d}\sigma\]
\item
  \textbf{保序性:} 若 \(f,g\in R(D)\),且当 \((x,y)\in D\)
  时,\(f(x,y)\leq g(x,y)\),则

  \[\iint_D f(x,y)\,\mathrm{d}\sigma \leq \iint_D g(x,y)\,\mathrm{d}\sigma\]

  \begin{itemize}
  \item
    \textbf{推论1:} 若 \(f\in R(D)\) ,则
    \(\vert f(x,y) \vert \in R(D)\),且
  \end{itemize}

  \[\left\vert \iint_D f(x,y)\,\mathrm{d}\sigma \right\vert \leq \iint_D \vert f(x,y) \vert \,\mathrm{d}\sigma\]

  \begin{itemize}
  \item
    \textbf{推论2:} 若 \(f\in R(D)\),且当 \((x,y)\in D\) 时,
    \(m\leq f(x,y) \leq M\),则
  \end{itemize}

  \[mA_D \leq \iint_D f(x,y)\,\mathrm{d}\sigma \leq MA_D\]

  \begin{itemize}
  \item
    \textbf{推论3:} 若 \(f\in R(D)\),且当 \((x,y)\in D\)
    时,\(f(x,y) \geq 0\),则
  \end{itemize}

  \[\iint_D f(x,y)\,\mathrm{d}\sigma \geq 0\]
\item
  \textbf{积分中值定理:} 若 \(f(x,y)\in C(D)\) ,\(g(x,y)\in R(D)\),且
  \(g\) 在 \(D\) 上不变号,则 \(\exists \xi , \eta \in D\),使得

  \[\iint_D f(x,y)g(x,y)\,\mathrm{d}\sigma = f(\xi ,\eta )\iint_D g(x,y)\,\mathrm{d}\sigma\]

  \begin{itemize}
  \item
    \textbf{推论:} 若 \(f(x,y)\in C(D)\),则存在
    \((\xi , \eta )\in D\),使得
  \end{itemize}

  \[\iint_D f(x,y)\,\mathrm{d}\sigma = f(\xi ,\eta )A_D\]

  称 \(f(\xi,\eta) = \frac{\iint_D f\,\mathrm{d}\sigma}{A_D}\) 为函数
  \(f(x,y)\) 在有界闭区域 \(D\) 上的\textbf{平均值}
\end{enumerate}

\section{二重积分的计算}

\subsection{直角坐标系下的计算}

当二重积分存在时,可利用平行于坐标轴的直线来划分积分区域 \(D\),此时,面积元素

\[\mathrm{d}\sigma = \mathrm{d}x\mathrm{d}y\]

故二重积分在直角坐标系下可表示为

\[\iint_D f(x,y)\,\mathrm{d}\sigma = \iint_D f(x,y)\,\mathrm{d}x\mathrm{d}y\]

\subsubsection{\(x\) 型正则区域}

\[D = \left\{(x,y)\bigg| \varphi_1(x)\leq y\leq \varphi_2(x),a\leq x\leq b\right\}\]

化为先 \(y\) 后 \(x\) 的二次积分

\[
\begin{aligned}
    \iint_D f(x,y)\,\mathrm{d}x\mathrm{d}y &= \int_a^b\left[\int_{\varphi_1(x)}^{\varphi_2(x)} f(x,y)\,\mathrm{d}y\right]\mathrm{d}x \\
    &\equiv \int_a^b f(x,y)\,\mathrm{d}x\mathrm{d}y
\end{aligned}
\]

\subsubsection{\(y\) 型正则区域}

\[D = \left\{(x,y)\bigg| \varphi_1(y)\leq x\leq \varphi_2(y),c\leq y\leq d\right\}\]

化为先 \(x\) 后 \(y\) 的二次积分

\[
\begin{aligned}
    \iint_D f(x,y)\,\mathrm{d}x\mathrm{d}y &= \int_c^d\left[\int_{\varphi_1(y)}^{\varphi_2(y)} f(x,y)\,\mathrm{d}x\right]\mathrm{d}y \\
    &\equiv \int_c^d f(x,y)\,\mathrm{d}x\mathrm{d}y
\end{aligned}
\]

\subsubsection{一般区域的二重积分}

分割成若干个正则子区域,利用积分区域可加性,分别在子区域上积分后求和

\begin{remark}
直角坐标计算二重积分的步骤

\begin{enumerate}
\def\labelenumi{\arabic{enumi}.}
\item
  \textbf{画出积分区域} \(D\) 的草图,并\textbf{确定类型}
\item
  按照所确定的类型\textbf{表示区域} \(D\)
\item
  \textbf{化二重积分为二次积分}(注意上下限)
\item
  \textbf{计算}二重积分
\end{enumerate}
\end{remark}

\subsection{极坐标系下的计算}

当积分区域的边界曲线或被积函数用极坐标表示较为简单时,可用极坐标来计算二重积分.

面积元素 \(\Delta \sigma\) 在极坐标下为

\[\boxed{\Delta\sigma= r\mathrm{d}r\mathrm{d}\theta}\]

从直角坐标到极坐标时的二重积分变换公式为

\[\iint_D f(x,y)\,\mathrm{d}\sigma = \iint_D f(r\cos\theta,r\sin\theta)r\,\mathrm{d}r\mathrm{d}\theta\]

\[\left\{\left(r,\theta\right)\left \vert r_1(\theta)\leq r \leq r_2(\theta) , \alpha \leq \theta \leq \beta \right.\right\}\]

则

\[\iint_D f(x,y)\,\mathrm{d}\sigma = \int_\alpha^\beta\, \mathrm{d}\theta\int_{r_1(\theta)}^{r_2(\theta)} f(r\cos\theta,r\sin\theta)r\,\mathrm{d}r\]

\subsection{二重积分的变量代换}

\begin{theorem}
    设变换 \(\displaystyle T :\begin{cases}x=x(u,v)\\y=y(u,v)\end{cases}\) 有连续偏导数,且满足 \(\displaystyle J = \frac{\partial (x,y)}{\partial (u,v)}\coloneqq \begin{vmatrix}x_u ,x_v\newline y_u,y_v\end{vmatrix} \left(\text{Jacobi 行列式}\right)\neq 0\),而 \(f(x,y)\in C(D)\),则

    \[\iint_D f(x,y)\,\mathrm{d}\sigma = \iint_{D^*} f(x(u,v),y(u,v))\left\vert J \right\vert \,\mathrm{d}u\mathrm{d}v\]
\end{theorem}

\begin{itemize}
\item
  在定理条件下,变换 \(T\) 一定存在逆变换
  \(T^{-1}:\begin{cases}u=u(x,y)\\v=v(x,y)\end{cases}\),且
  \(\frac{\partial (u,v)}{\partial (x,y)}\cdot \frac{\partial (x,y)}{\partial (u,v)} = 1\)

  有时,借助此式求 \(J\) 较为简单
\item
  当 Jacobi 行列式仅在区域 \(D^*\) 内个别点上或个别曲线上为 \(0\)
  时,定理结论仍成立
\item
  在广义极坐标
  \(\begin{cases}x=ar\cos\theta\\y=br\sin\theta\end{cases}\)
  下,\(J = abr\)
\end{itemize}

\section{三重积分}

\subsection{三重积分的定义}

\begin{definition}
    设 \(\Omega\) 是 \(\mathbb{R}^3\) 中的有界闭区域,三元函数 \(f(x,y,z)\) 在 \(\Omega\) 上有界,\(I\) 为实数. 若将 \(\Omega\) 任意分割成 \(n\) 个小区域 \(\Delta \Omega_1,\Delta\Omega_2,\ldots,\Delta\Omega_n\),任取 \(\left(\varepsilon_i,\eta_i,\xi_i \right)\in\Delta\Omega_i\left(i=1,2,\ldotp,n\right)\),作和 \(\displaystyle\sum_{i=1}^nf\left(\varepsilon_i,\eta_i,\xi_i \right)\Delta V_i\),(\(\Delta V_i\) 是 \(\Delta\Omega_i\) 的体积),总有

    \[\lim_{\lambda\to 0}\sum_{i=1}^nf\left(\varepsilon_i,\eta_i,\xi_i \right)\Delta V_i = I\]

    其中 \(\lambda = \max_{1\leq i\leq n}\{d_i\}\),\(d_i\) 是 \(\Delta\Omega_i\) 的直径,则称函数 \(f(x,y,z)\) 在 \(\Omega\) 上\textbf{可积},记为 \(f\in R(\Omega)\); \(I\) 称为 \(f(x,y,z)\) 在 \(\Omega\) 上的\textbf{三重积分},记作

    \[\iiint_\Omega f(x,y,z)\,\mathrm{d}V\]

    其中 \(V_\Omega\) 是区域 \(\Omega\) 的体积
\end{definition}

\begin{itemize}
\item
  若 \(f(x,y,z)\) 表示占有三维空间区域 \(\Omega\) 的物体的体密度函数,则 \(\displaystyle\iiint_\Omega f(x,y,z)\,\mathrm{d}V\) 给出了物体的\textbf{质量}
\item
  类似二重积分,三重积分具有线性性,区域可加性,保序性以及推论和积分中值定理,并且有 \(\displaystyle\iiint_\Omega \,\mathrm{d}V = V_\Omega\)
\end{itemize}

\subsection{直角坐标系下的计算}

直角坐标系下,由于 \(\mathrm{d}V=\mathrm{d}x\mathrm{d}y\mathrm{d}z\),故

\[\iiint_\Omega f(x,y,z)\,\mathrm{d}V = \iiint_\Omega f(x,y,z)\,\mathrm{d}x\mathrm{d}y\mathrm{d}z\]

\subsubsection{柱线法(坐标面投影法)}

设 \(\Omega\) 是以曲面 \(z=z_1(x,y)\) 为底,曲面
\(x=x_2(x,y)\),而侧面是母线平行于 \(z\) 轴的柱体所围成的区域

设 \(\Omega\) 在 \(xOy\) 面上的投影区域为 \(D_1\) ,则 \(\Omega\)
可表示为

\[\Omega = \left\{(x,y,z)\bigg| (x,y)\in D_1 , z_1(x,y)\leq z\leq z_2(x,y)(x,y)\in D\right\}\]

则物体总质量为

\[\iint_D\left(\int_{z_1(x,y)}^{z_2(x,y)}f(x,y,z)\,\mathrm{d}z\right) \,\mathrm{d}x\mathrm{d}y\]

故有

\[\iiint_\Omega f(x,y,z)\,\mathrm{d}V = \iint_{D_1} \,\mathrm{d}x\mathrm{d}y\int_{z_1(x,y)}^{z_2(x,y)} f(x,y,z)\,\mathrm{d}z\]

\subsubsection{截面法(坐标轴投影法)}

设区域 \(\Omega\) 在 \(z\) 轴上的投影区间为 \([h_1,h_2]\),即 \(\Omega\)
介于平面 \(z=h_1\) 和 \(z=h_2\) 之间,过 \(z\) 处且垂直于 \(z\)
轴的平面截 \(\Omega\) 得截面区域 \(D_z\),则 \(\Omega\) 可表示为

\[\Omega = \left\{(x,y,z)\bigg| h_1\leq z\leq h_2 , (x,y)\in D_z\right\}\]

物体总质量为

\[\int_{h_1}^{h_2}\left(\iint_{D_z} f(x,y,z)\,\mathrm{d}x\mathrm{d}y\right) \,\mathrm{d}z\]

故有

\[\iiint_\Omega f(x,y,z)\,\mathrm{d}V = \int_{h_1}^{h_2}\,\mathrm{d}z\iint_{D_z} f(x,y,z)\,\mathrm{d}x\mathrm{d}y\]

\subsection{三重积分的变量代换}

\begin{theorem}
    设变换 \(T:\begin{cases}x=x(u,v,w)\\y=y(u,v,w)\\z=z(u,v,w)\end{cases}\)
有连续偏导数,且满足
\(J=\frac{\partial (x,y,z)}{\partial (u,v,w)}\neq 0\),而
\(f(x,y,z)\in C(\Omega)\),则

\[\iiint_\Omega f(x,y,z)\,\mathrm{d}V = \iiint_{\Omega^*} f(x(u,v,w),y(u,v,w),z(u,v,w))\left\vert J \right\vert \,\mathrm{d}u\mathrm{d}v\mathrm{d}w\]
\end{theorem}

\subsubsection{柱面坐标系下的计算}

柱面坐标系,实际上就是将 \(x,y\) 坐标转换为极坐标

\[
\begin{cases}
x = r\cos\theta\\
y = r\sin\theta\\
z = z
\end{cases}
\]

其 Jacobi 行列式为

\[J = \frac{\partial (x,y,z)}{\partial (r,\theta,z)} = \begin{vmatrix}
\cos\theta & -r\sin\theta & 0\\
\sin\theta & r\cos\theta & 0\\
0 & 0 & 1 
\end{vmatrix} = r\]

则柱面积分积分公式为

\[\iiint_\Omega f(x,y,z)\,\mathrm{d}V = \iiint_{\Omega^*} f(r\cos\theta,r\sin\theta,z)r\,\mathrm{d}r\mathrm{d}\theta\mathrm{d}z\]

\subsubsection{球面坐标系下的计算}

球面坐标系,实际上就是将 \(x,y,z\) 坐标转换为球坐标

\[\begin{cases}
x = \rho\sin\varphi\cos\theta\\
y = \rho\sin\varphi\sin\theta\\
z = \rho\cos\varphi
\end{cases}\]

其 Jacobi 行列式为

\[J = \frac{\partial (x,y,z)}{\partial (\rho,\varphi,\theta)} = \begin{vmatrix}
\sin\varphi\cos\theta & \rho\cos\varphi\cos\theta & -\rho\sin\varphi\sin\theta\\
\sin\varphi\sin\theta & \rho\cos\varphi\sin\theta & \rho\sin\varphi\cos\theta\\
\cos\varphi & -\rho\sin\varphi & 0
\end{vmatrix} = \rho^2\sin\varphi\]

则球面积分积分公式为

\[\iiint_\Omega f(x,y,z)\,\mathrm{d}V = \iiint_{\Omega^*} f(\rho\sin\varphi\cos\theta,\rho\sin\varphi\sin\theta,\rho\cos\varphi)\rho^2\sin\varphi\,\mathrm{d}\rho\mathrm{d}\varphi\mathrm{d}\theta\]

\section{重积分的应用}

\subsection{重积分的几何应用}

\subsubsection{平面图形的面积}

\[A(D) = \iint_D \,\mathrm{d}\sigma = \iint_D \,\mathrm{d}x\mathrm{d}y\]

\subsubsection{立体的体积}

\[V(\Omega) = \iiint_\Omega \,\mathrm{d}V = \iiint_\Omega \,\mathrm{d}x\mathrm{d}y\mathrm{d}z\]

\subsubsection{曲面的面积}

空间曲面 \(S:z=f(x,y),(x,y)\in D\).

则曲面 \(S\) 的面积为

\[A(S) = \iint_D \sqrt{1+z_x^2+z_y^2}\,\mathrm{d}x\mathrm{d}y\]

\subsection{重积分的物理应用}

\subsubsection{质心}

体密度为 \(\rho(x, y)\) 的物体占据空间 \(\Omega\),其质心坐标为

\[
\begin{cases}
  \displaystyle\bar{x} = \frac{\iiint_\Omega x\rho(x,y,z)\,\mathrm{d}V}{\iiint_\Omega\mu(x,y,z)\,\mathrm{d}V} \\
  \displaystyle\bar{y} = \frac{\iiint_\Omega y\rho(x,y,z)\,\mathrm{d}V}{\iiint_\Omega\mu(x,y,z)\,\mathrm{d}V}\\
  \displaystyle\bar{z} = \frac{\iiint_\Omega z\rho(x,y,z)\,\mathrm{d}V}{\iiint_\Omega\mu(x,y,z)\,\mathrm{d}V}
\end{cases}
\]

\subsubsection{转动惯量}\label{ux8f6cux52a8ux60efux91cf}

设物体的密度为 \(\rho(x,y,z)\),则物体分别关于 \(x\),\(y\),\(z\)
轴的转动惯量为

\[\begin{cases}
  I_x = \iiint_\Omega \rho(x,y,z)(y^2+z^2)\,\mathrm{d}V\\
  I_y = \iiint_\Omega \rho(x,y,z)(x^2+z^2)\,\mathrm{d}V\\
  I_z = \iiint_\Omega \rho(x,y,z)(x^2+y^2)\,\mathrm{d}V
\end{cases}\]

\subsubsection{引力}

\[\begin{aligned}
  &\mathrm{d}\vec{F} = G\frac{m_0\mathrm{d}m}{r^3}\vec{r}\\
  =& G\frac{m_0\rho(x,y,z)\mathrm{d}V}{r^3}\cdot\left(x-x_0,y-y_0,z-z_0\right)\\
  =&\left(\mathrm{d}F_x,\mathrm{d}F_y,\mathrm{d}F_z\right)
\end{aligned}\]

\chapter{曲线积分和曲面积分}

\section{第一类曲线积分和曲面积分}

\subsection{第一类曲线积分的概念}

\begin{definition}
    设 \(C\) 是 \(xOy\) 面上的一条光滑曲线弧,函数 \(f(x,y)\) 是定义在 \(C\)
上的有界函数,在 \(C\) 上任意插入分点
\(A = A_0,A_1,\ldots,A_{n-1},A_n=B\),将其分成 \(n\) 个小弧段,记第
\(i\) 个小弧段的弧长为 \(\Delta s_i\),在第 \(i\) 个小段上任取点
\((\epsilon_i,\eta_i)\),和式
\(\displaystyle\sum_{i=1}^{+\infty}f(\epsilon_i,\eta_i)\Delta s_i\),当
\(\displaystyle\lambda=\max_{1\leq i\leq n}\{\Delta s_i\}\to 0\)
时,有确定的极限值 \(I\),即

\[\lim_{\lambda\to 0}\sum_{i=1}^nf(\epsilon_i,\eta_i)\Delta s_i = I\]

则称函数 \(f(x,y)\) 在曲线 \(C\) 上\textbf{可积},并将此极限值 \(I\)
称为函数 \(f(x,y)\) 在曲线 \(C\) 上的\textbf{第一类曲线积分},记作
\(\displaystyle\int_Cf(x,y)\,\mathrm{d}s\),即

\[I = \int_C f(x,y)\,\mathrm{d}s = \lim_{\lambda\to 0}\sum_{i=1}^nf(\epsilon_i,\eta_i)\Delta s_i\]
\end{definition}

\begin{itemize}
\item
  第一类曲线积分的几何含义为柱面的面积
\item
  \(\displaystyle\int_C\,\mathrm{d}s = \displaystyle\int_c1\,\mathrm{d}s=s_C\)
\item
  若 \(C\) 是封闭曲线,即 \(C\) 的二端点重合,则记第一类曲线积分为
  \(\displaystyle\oint_Cf(x,y)\,\mathrm{d}s\)
\end{itemize}

\subsection{第一类曲线积分的性质}

\subsubsection{与曲线方向无关}

若曲线 \(C\) 的端点为 \(A\) 和 \(B\),\(f(x,y)\) 在曲线 \(C\) 上可积,则

\[\int_{\widehat{AB}} f(x,y)\,\mathrm{d}s=\oint_{\widehat{BA}} f(x,y)\,\mathrm{d}s\]

\subsubsection{线性性}

若 \(f,g\) 在曲线 \(C\) 上可积,\(\alpha , \beta\) 是任意常数,则
\(\alpha f + \beta g\) 在曲线 \(C\) 上可积,且

\[\int_C (\alpha f + \beta g)\,\mathrm{d}s = \alpha \int_C f(x,y)\,\mathrm{d}s + \beta \int_C g(x,y)\,\mathrm{d}s\]

\subsubsection{路径可加性}

若曲线 \(C\) 由两段光滑曲线 \(C_1\) 和 \(C_2\) 首尾连接而成,则
\(f(x,y)\) 在曲线 \(C\) 上可积,等价于 \(f(x,y)\) 在曲线 \(C_1\) 和
\(C_2\) 上均可积,且

\[\int_C f(x,y)\,\mathrm{d}s = \int_{C_1} f(x,y)\,\mathrm{d}s + \int_{C_2} f(x,y)\,\mathrm{d}s\]

\subsubsection{中值定理}

设函数 \(f\) 在光滑曲线 \(C\) 上连续,则
\(\exists (\epsilon,\eta)\in C\),使得

\[\int_C f(x,y)\,\mathrm{d}s = f(\epsilon,\eta)\cdot s_C\]

其中 \(s_C\) 是曲线段 \(C\) 的长度

\subsection{一类曲线积分的计算}

设函数 \(f(x,y)\) 在光滑曲线 \(C\) 上连续,\(C\) 的参数方程为
\(\begin{cases}x=x(t)\\y=y(t)\end{cases}\),\(t\in [a,b]\),满足
\(x'(t)\),\(y'(t)\) 连续,且 \(x'(t)^2 + y'(t)^2 \neq 0\),则

\[\int_C f(x,y)\,\mathrm{d}s = \int_\alpha^\beta f(x(t),y(t))\sqrt{x'(t)^2 + y'(t)^2}\,\mathrm{d}t\]

\begin{itemize}
\item
  右端积分限应 \(\alpha < \beta\)
\item
  当曲线 \(C\) 形式为 \(y=y(x)\),\(x\in [a,b]\) 时
  \[\displaystyle\int_Cf(x,y)\,\mathrm{d}s =\int_a^bf(x,y(x))\sqrt{1+y'^2(x)}\,\mathrm{d}x\]
\item
  当曲线 \(C\) 为极坐标 \(r=r(\theta)\),\(\theta\in [\alpha,\beta]\)
  时
  \[\displaystyle\int_Cf(x,y)\,\mathrm{d}s = \int_\alpha^\beta f(r(\theta)\cos\theta,r(\theta)\sin\theta)\sqrt{r(\theta)^2+r'^2(\theta)}\,\mathrm{d}\theta\]
\end{itemize}

\section{第二类曲线积分与曲面积分}

\subsection{第二类曲线积分的概念}

\begin{definition}
    设 \(C\) 为平面光滑定向曲线(\(A\rightarrow B\)),且向量值函数
\(\vec{F}(x,y) = R(x,y)\vec{i}+Q(x,y)\vec{j}\) 在 \(C\)
上有界,\(\vec{e}_\tau\) 为 \(C\) 上点 \((x,y)\)
处于定向一致的单位切向量,若

\[\int_C \vec{F}(x,y) \cdot \vec{e}_\tau \,\mathrm{d}s\]

存在,则称为\textbf{向量值函数 \(\vec{F}\) 在定向曲线 \(C\)
上的曲线积分或第二类曲线积分}
\end{definition}

若 \(\vec{e}_\tau(x,y) = (\cos\alpha,\cos\beta)\),则

\[\begin{aligned}
  \int_C \vec{F}(x,y) \cdot \vec{e}_\tau \,\mathrm{d}s &= \int_C P(x,y)\cos\alpha + Q(x,y)\cos\beta \,\mathrm{d}s \\
  &= \int_C P(x,y)\cos\alpha \,\mathrm{d}s + \int_C Q(x,y)\cos\beta \,\mathrm{d}s \\
  &= \int_C P(x,y)\,\mathrm{d}x +  Q(x,y)\,\mathrm{d}y
\end{aligned}\]

这是对坐标的曲线积分

记 \(\vec{r} = (x,y)\),则
\(\mathrm{d}\vec{e} = \vec{e}_\tau \,\mathrm{d}s\)
称为\textbf{定向弧微分}

从而有向量形式的第一类曲线积分

\[\int_C \vec{F}(x,y) \cdot \mathrm{d}\vec{e} = \int_C \vec{F}\cdot \mathrm{d}\vec{r}\]

\subsubsection{第二类曲线积分的性质}

第二类曲线积分与\textbf{曲线方向有关},即

\[\int_{\widehat{AB}} \vec{F}(x,y) \cdot \mathrm{d}\vec{r} = -\oint_{\widehat{BA}} \vec{F}(x,y) \cdot \mathrm{d}\vec{r}\]

此外线性性与对定向积分路径的可加性等仍然成立

\subsubsection{第二类曲线积分的计算}

若曲线 \(C\) 为
\(\begin{cases}x=x(t)\\y=y(t)\end{cases}\),\(t:\alpha\rightarrow \beta\)

起点 \(A\) 对应 \(\alpha\),终点 \(B\) 对应 \(\beta\)

考察
\(\displaystyle\int_C P\,\mathrm{d}x + Q\,\mathrm{d}y = \int_C \vec{F}\cdot\vec{e}_\tau \,\mathrm{d}s\),沿曲线
\(C\) 有
\(\vec{F} = \left(P\left(x(t),y(t)\right),Q\left(x(t),y(t)\right)\right)\),则

\[\int_C P\,\mathrm{d}x + Q\,\mathrm{d}y = \int_\alpha^\beta P\left(x(t),y(t)\right)\,\mathrm{d}x(t) + Q\left(x(t),y(t)\right)\,\mathrm{d}y(t)\]

\subsection{第二类曲面积分的概念}

\subsubsection{双侧曲面}

\begin{definition}
    若点 \(P\) 沿曲面 \(S\)
上任何不越过曲面边界的连续闭曲线移动后回到起始位置时,法向量 \(\vec{n}\)
保持原来的指向,则称 \(S\) 为\textbf{双侧曲面}
\end{definition}

典型的,Mobius 面不是双侧曲面

选定双侧曲面 \(S\) 一侧为正向,称为\textbf{正侧},记为 \(S^+\)
,其相反测记作 \(S^-\)

\subsubsection{双侧曲面定侧}

若
\(S:z=z(x,y)\),\((x,y)\in D_{xy}\),\(\vec{n}_0=\left(\cos\alpha,\cos\beta,\cos\gamma\right)=\pm\frac{\left(-z_x,-z_y,1\right)}{\sqrt{1+z_x^2+z_y^2}}\)

若选取
\(\vec{n}_0=\left(\cos\alpha,\cos\beta,\cos\gamma\right)=\frac{\left(-z_x,-z_y,1\right)}{\sqrt{1+z_x^2+z_y^2}}\),则说明
\(\cos\gamma > 0\),选取了曲面的上侧

一般的

\[\begin{cases}
  \cos\alpha >0 \Leftrightarrow \text{前侧},\cos\alpha < 0 \Leftrightarrow \text{后侧}\\
  \cos\beta >0 \Leftrightarrow \text{右侧},\cos\beta < 0 \Leftrightarrow \text{左侧}\\
  \cos\gamma >0 \Leftrightarrow \text{上侧},\cos\gamma < 0 \Leftrightarrow \text{下侧}
\end{cases}\]

习惯上选取曲面片的上侧为 \(S^+\);对于封闭曲面,选取外侧为 \(S^+\)

对于向量值函数 \(\vec{F} = (P,Q,R)\)

\[\int_C \vec{F}\cdot\,\mathrm{d}S = \int_c P\,\mathrm{d}x\mathrm{d}y + Q\,\mathrm{d}y\mathrm{d}z + R\,\mathrm{d}z\mathrm{d}x\]

\subsubsection{第二类曲面积分的性质}

第二类曲面积分与在曲面的哪一侧积分有关

\[\iint_{S^+}P\,\mathrm{d}x\mathrm{d}y + Q\,\mathrm{d}y\mathrm{d}z + R\,\mathrm{d}z\mathrm{d}x = -\iint_{S^-}P\,\mathrm{d}x\mathrm{d}y + Q\,\mathrm{d}y\mathrm{d}z + R\,\mathrm{d}z\mathrm{d}x\]

此外第二类曲面积分也具有线性性和可加性等性质

\subsection{第二类曲面积分的计算}

\subsubsection{合一投影法}

\[\iint_{S^+}P\,\mathrm{d}x\mathrm{d}y + Q\,\mathrm{d}y\mathrm{d}z + R\,\mathrm{d}z\mathrm{d}x = \iint_{D_{xy}} \left(-Pz_x-Qz_y+R\right)\,\mathrm{d}x\mathrm{d}y\]

\subsubsection{分面投影法}

分
\(P\,\mathrm{d}x\mathrm{d}y\),\(Q\,\mathrm{d}y\mathrm{d}z\),\(R\,\mathrm{d}z\mathrm{d}x\)
三个部分进行积分

常在部分曲面垂直坐标轴时进行

\subsubsection{公式法}

常用于参数方程确定的曲面

设 \(S:\vec{r} = \left(x(u,v),y(u,v),z(u,v)\right)\),其中
\((u,v)\in D_{xy}\),则

\[\iint_{S^+}\vec{F}\cdot\,\mathrm{d}\vec{S} = \iint_{D_{uv}}\vec{F}\cdot\left(\vec{r}_u\times\vec{r_v}\right)\,\mathrm{d}u\mathrm{d}v\]

\section{Green 公式及其应用}

\subsection{Green 公式}

\subsubsection{连通区域及其边界方向}

设 \(D\) 为平面区域, 若 \(D\) 内的任意一条闭曲线所围的区域都落在 \(D\)
内, 则称 \(D\) 是单连通的, 否则称 \(D\) 为复连通的

当点沿区域边界朝一个方向前进时, 区域总在它的左侧,
则将此方向规定为边界曲线 \(C\) 的正向,记为 \(C^+\), 与 \(C^+\)
相反方向为 \(C^-\)

\subsubsection{Green 公式}

\begin{theorem}
    设有界闭区域 \(D\) 由分段光滑曲线 \(C\) 围成,函数 \(P(x, y)\),
\(Q(x, y)\) 在 \(D\) 上有一阶连续偏导数, 则

\[\oint_{C^+} P\,\mathrm{d}x + Q\,\mathrm{d}y = \iint_D \left(\frac{\partial Q}{\partial x} - \frac{\partial P}{\partial y}\right)\,\mathrm{d}x\mathrm{d}y\]
\end{theorem}

\begin{itemize}
\item
  对于复连通区域 \(D\),Green 公式仍然成立,但需将 \(C\)
  分成若干个单连通区域 \(D_i\),并对每个区域应用 Green 公式
\item
  公式也可以记为
  \(\displaystyle\oint_{C^+} P\,\mathrm{d}x + Q\,\mathrm{d}y = \iint_D \begin{vmatrix}\frac{\partial}{\partial x}&\frac{\partial}{\partial y}\newline P&Q\end{vmatrix}\,\mathrm{d}x\mathrm{d}y\)
\end{itemize}

\subsubsection{Green 公式的向量形式}

\subsection{线积分与路径无关的条件}

\begin{definition}
    设 \(P(x,y)\),\(Q(x,y)\) 在区域 \(D\) 内连续,若对 \(D\) 内任意两点
\(A\),\(B\) 以及 \(D\) 内连接 \(A,B\) 的任意二分段光滑曲线
\(C_1\),\(C_2\),均有

\[\int_{C_1} P\,\mathrm{d}x +Q\,\mathrm{d}y = \int_{C_2} P\,\mathrm{d}x +Q\,\mathrm{d}y\]

则称曲线积分 \(\displaystyle\int_C P\,\mathrm{d}x +Q\,\mathrm{d}y\) 在
\(D\) 内\textbf{与路径无关}
\end{definition}

\begin{theorem}
    设函数 \(P\),\(Q\) 在\textbf{单连通}区域 \(D\)
上有连续偏导数,则下述四命题等价

\begin{enumerate}
\def\labelenumi{\arabic{enumi}.}
\item
  在 \(D\) 内的任一条分段光滑闭曲线 \(C\) 上,有
  \(\displaystyle \int_C P\,\mathrm{d}x +Q\,\mathrm{d}y = 0\)
\item
  曲线积分 \(\displaystyle \int_C P\,\mathrm{d}x +Q\,\mathrm{d}y\) 在
  \(D\) 内与路径无关
\item
  存在 \(D\) 上的可微函数 \(u(x,y)\) 使得
  \(\displaystyle \mathrm{d}u = P\,\mathrm{d}x +Q\,\mathrm{d}y\),此时称
  \(u(x,y)\) 为 \(\displaystyle P\,\mathrm{d}x +Q\,\mathrm{d}y\)
  的一个\textbf{原函数}
\item
  \(\displaystyle \frac{\partial Q}{\partial x} = \frac{\partial P}{\partial y}\)
  在 \(D\) 内恒成立
\end{enumerate}
\end{theorem}

\subsection{全微分求积与全微分方程}

设函数 \(P\),\(Q\) 在单连通区域 \(D\) 上有连续偏导数,且
\(\displaystyle \frac{\partial Q}{\partial x} = \frac{\partial P}{\partial y}\)
,则 \(\displaystyle P\,\mathrm{d}x +Q\,\mathrm{d}y\) 为某函数 \(u\)
的全微分,且取定 \((x_0,y_0)\in D\)

\[u(x,y) = u(x_0,y_0) + \int_{(x_0,y_0)}^{(x,y)} P\,\mathrm{d}x +Q\,\mathrm{d}y,\quad (x,y)\in D\]

从而全体函数为 \(u(x,y) + C\)

称求 \(P\,\mathrm{d}x +Q\,\mathrm{d}y\)
的原函数的过程为\textbf{全微分求积}

若 \(P\,\mathrm{d}x +Q\,\mathrm{d}y\) 是某二元函数的全微分,称方程

\[P(x,y)\,\mathrm{d}x +Q(x,y)\,\mathrm{d}y = 0\]

为\textbf{全微分方程}

求出一个原函数 \(u(x,y)\),则方程的通解为 \(u(x,y) = C\),其中 \(C\)
是任意常数

\section{Gauss 公式和 Strokes 公式}

\subsection{Gauss 公式}

\begin{theorem}
    设函数 \(P(x,y,z)\),\(Q(x,y,z)\),\(R(x,y,z)\) 在空间有界闭区域
\(\Omega\) 上有连续偏导数,\(\Omega\) 的边界时光滑或分片光滑的闭曲面
\(\Sigma\),则

\[\oiint_{\Sigma^+} P\,\mathrm{d}x\mathrm{d}y + Q\,\mathrm{d}y\mathrm{d}z + R\,\mathrm{d}z\mathrm{d}x = \iiint_\Omega \left(\frac{\partial P}{\partial x} + \frac{\partial Q}{\partial y} + \frac{\partial R}{\partial z}\right)\,\mathrm{d}x\mathrm{d}y\mathrm{d}z\]
\end{theorem}

\begin{itemize}
\item
  令 \(P=\frac{x}{3}\),\(Q=\frac{y}{3}\),\(R=\frac{z}{3}\),则可导出
  \(\displaystyle V_\Omega = \frac{1}{3}\oiint_{\Sigma^+} x\,\mathrm{d}y\mathrm{d}z + y\,\mathrm{d}z\mathrm{d}x + z\,\mathrm{d}x\mathrm{d}y\),即体积公式
\item
  使用 Gauss 公式时,注意 \(\Sigma^+\) 的方向应与 \(\Omega\) 的外侧一致
\end{itemize}

\subsubsection{向量形式的 Gauss 公式}

\[\oiint_{\Sigma^+} \vec{F}\cdot\,\mathrm{d}\vec{S} = \iiint_\Omega \nabla\cdot\vec{F}\,\mathrm{d}x\mathrm{d}y\mathrm{d}z\]

\subsection{通量和散度}

\subsubsection{通量}

若给定向量场

\[\vec{F} = \left(P(x,y,z),Q(x,y,z),R(x,y,z)\right)\]

则称曲面积分

\[\Phi = \oiint_{\Sigma^+} \vec{F}\cdot\,\mathrm{d}\vec{S} = \oiint_{\Sigma^+} P\,\mathrm{d}x\mathrm{d}y + Q\,\mathrm{d}y\mathrm{d}z + R\,\mathrm{d}z\mathrm{d}x\]

为向量场 \(\vec{F}\) 在通过定侧曲面 \(\Sigma^+\) 的\textbf{通量}

\subsubsection{散度}

称

\[\mathrm{div}\vec{F} = \nabla\cdot\vec{F}=\frac{\partial P}{\partial x} + \frac{\partial Q}{\partial y} + \frac{\partial R}{\partial z}\]

为向量场 \(\vec{F}\) 的\textbf{散度}

则 Gauss 公式可写为

\[\Phi =\oiint_{\Sigma^+} \vec{F}\cdot\,\mathrm{d}\vec{S} = \iiint_\Omega \mathrm{div}\vec{F}\,\mathrm{d}V\]

\chapter{级数}

\section{数项级数}

\subsection{数项级数的概念}

\begin{definition}
给定数列 \(\{a_n\}\) ,和式

\[\sum_{n=1}^{\infty} a_n = a_1 + a_2 + \cdots + a_n + \cdots\]

称为 \textbf{(无穷)极数} ,\(a_n\)
称为级数的\textbf{通项}(或\textbf{一般项})
\end{definition}

\begin{itemize}
\item
  \(\displaystyle S_n = \sum_{k=1}^n a_k\) 称为级数
  \(\displaystyle \sum_{n=1} a_n\) 的前 \(n\) 项\textbf{部分和}
\item
  \(\displaystyle\sum_{k=n+1}^\infty a_k\) 称为级数
  \(\displaystyle \sum_{n=1} a_n\) 的\textbf{余项级数}
\end{itemize}

\begin{definition}
\begin{itemize}
\item
  若级数 \(\displaystyle\sum_{n=1}^\infty a_n\) 的部分和数列 \(\{S_n\}\)
  收敛,且 \(\displaystyle \lim_{n\to\infty}S_n = S\),则称级数
  \(\displaystyle \sum_{n=1}^\infty a_n\) \textbf{收敛},\(S\) 称为级数
  \(\displaystyle \sum_{n=1}^\infty a_n\) 的\textbf{和},记作
  \(\displaystyle \sum_{n=1}^\infty a_n = S\)
\item
  若部分和数列 \(\{S_n\}\) 发散,则称级数
  \(\displaystyle \sum_{n=1}^\infty a_n\) \textbf{发散}
\end{itemize}
\end{definition}

\begin{remark}
    常用结论:

\textbf{等比数列}
\(\displaystyle \sum_{n=1}^\infty aq^{n-1}\begin{cases}\text{收敛于}\frac{a}{1-q}  ,\vert q \vert < 1\newline \text{发散}  ,\vert q \vert \geq 1\end{cases}\)
\end{remark}

\subsection{数项级数的概念}

\subsection{数项级数的基本性质}

\subsubsection{基本性质}

\begin{enumerate}
\def\labelenumi{\arabic{enumi}.}
\item
  若常数 \(\alpha \neq 0\),则级数
  \(\displaystyle \sum_{n=1}^\infty a_n\) 与级数
  \(\displaystyle \sum_{n=1}^\infty \alpha a_n\) 有相同敛散性
\item
  \textbf{线性性:} 若级数
  \(\displaystyle \sum_{n=1}^\infty a_n = S\),\(\displaystyle \sum_{n=1}^\infty b_n = T\),则
  \(\forall \alpha , \beta\in \mathbb{R}\),有
  \(\displaystyle \sum_{n=1}^\infty (\alpha a_n + \beta b_n) = \alpha S + \beta T\)
\item
  \textbf{可加性:}
  将级数增加、删减或改换\textbf{有限项},不改变级数的\textbf{敛散性}
\item
  \textbf{结合律:} 若级数收敛于
  \(S\),则将相邻若干项添加括号所成新级数仍收敛于 \(S\)

  \begin{itemize}
  \item
    其本质是部分和数列收敛于 \(S\),则子列均收敛于 \(S\)
  \item
    加括号后级数收敛 \(\nRightarrow\) 原级数收敛
  \item
    加括号后级数发散 \(\Rightarrow\) 原级数发散
  \end{itemize}
\end{enumerate}

\subsubsection{级数收敛的必要条件}

\begin{theorem}
    若 \(\displaystyle \sum_{n=1}^\infty a_n\) 收敛,则
\(\displaystyle \lim_{n\to\infty} a_n = 0\)
\end{theorem}

\begin{itemize}
\item
  若
  \(\displaystyle \lim_{n\to\infty} a_n \neq 0 \nRightarrow \displaystyle \sum_{n=1}^\infty a_n\)
  发散
\item
  若
  \(\displaystyle \lim_{n\to\infty} a_n = 0 \nRightarrow \displaystyle \sum_{n=1}^\infty a_n\)
  收敛,比如调和级数
\end{itemize}

\section{正项级数敛散性}

\subsection{正项级数}

\begin{definition}
    若级数 \(\displaystyle \sum_{n=1}^\infty a_n\) 满足
\(a_n >0 \quad \left(n\in\mathbb{N}^+\right)\),则称此级数为\textbf{正项级数}
\end{definition}

\begin{theorem}
    \textbf{收敛原理:}正项级数 \(\displaystyle \sum_{n=1}^\infty a_n\) 收敛
\(\Leftrightarrow\) 是其部分和数列 \(\{S_n\}\) 有上界,即
\(\exists M\in\mathbb{R},\forall n \in \mathbb{N}^+:S_n \leq M\)
\end{theorem}

\begin{remark}
    \(p\) 级数

\(\displaystyle \sum_{n=1}^\infty \frac{1}{n^p} \begin{cases}\text{收敛}  ,p>1\newline \text{发散} ,p\leq 1\end{cases}\)
\end{remark}

\subsection{正项级数敛散性判别法}

\subsubsection{比较判别法}

\begin{theorem}
    设正项级数
\(\displaystyle \sum_{n=1}^\infty a_n\),\(\displaystyle \sum_{n=1}^\infty b_n\)
满足 \(a_n \leq b_n \quad (\forall n \in \mathbb{N}^+)\) ,则
\(\displaystyle \sum_{n=1}^\infty b_n\) 收敛
\(\Rightarrow \displaystyle \sum_{n=1}^\infty a_n\)
收敛,\(\displaystyle \sum_{n=1}^\infty a_n\) 发散
\(\Rightarrow \displaystyle \sum_{n=1}^\infty b_n\) 发散
\end{theorem}

\begin{itemize}
\item
  条件 \(\forall n \in \mathbb{N}^+,a_n \leq b_n\) 可改为
  \(\exists N,C >0  ,\forall n\in \mathbb{N}^+,\forall n \geq N,a_n \leq Cb_n\)
\item
  使用该判别法时需要有参照级数,常选\textbf{等比级数}或 \textbf{\(p\)
  级数}作参照
\end{itemize}

\subsubsection{比较判别法(极限形式)}

\begin{theorem}
    正项级数
\(\displaystyle \sum_{n=1}^\infty a_n\),\(\displaystyle \sum_{n=1}^\infty b_n\)
满足 \(\displaystyle \lim_{n\to\infty} \frac{a_n}{b_n} = l\)

\begin{itemize}
\item
  当 \(0 <l<+\infty\) 时,\(\displaystyle \sum_{n=1}^\infty a_n\) 与
  \(\displaystyle \sum_{n=1}^\infty b_n\) 同敛散
\item
  当 \(l=0\) 时,\(\displaystyle \sum_{n=1}^\infty b_n\) 收敛
  \(\Rightarrow \displaystyle \sum_{n=1}^\infty a_n\) 收敛
\item
  当 \(l=+\infty\) 时,\(\displaystyle \sum_{n=1}^\infty b_n\) 发散
  \(\Rightarrow \displaystyle \sum_{n=1}^\infty a_n\) 发散
\end{itemize}
\end{theorem}

\begin{remark}
    通常使用 \(b_n = \frac{1}{n^p}\) 作为参照物,因为我们此时在分析无穷小
\(a_n\) 的阶
\end{remark}

\subsubsection{比值判别法(d'Alembert 判别法)}

\begin{theorem}
    若\textbf{正项级数} \(\displaystyle \sum_{n=1}^\infty a_n\) 满足
\(\displaystyle \lim_{n\to\infty} \frac{a_{n+1}}{a_n} = l\),则

\begin{itemize}
\item
  当 \(0\leq l<1\) 时,\(\displaystyle \sum_{n=1}^\infty a_n\) 收敛
\item
  当 \(l>1\) 时,\(\displaystyle \sum_{n=1}^\infty a_n\) 发散
\item
  当 \(l=1\) 时,判别法失效
\end{itemize}
\end{theorem}

\begin{remark}
    Stirling 公式:
\(n! \sim \left(\frac{n}{e}\right)^n \sqrt{2n\pi} \quad (n\to\infty)\)
\end{remark}

\begin{remark}
    当 \(a_n\) 是一些乘积构成或含 \(n!\) 时,可以考虑比值法
\end{remark}

\subsubsection{根值判别法(Cauchy 判别法)}

\begin{theorem}
若正项级数 \(\displaystyle \sum_{n=1}^\infty a_n\) 满足
\(\displaystyle \lim_{n\to\infty} \sqrt[n]{a_n} = l\),则

\begin{enumerate}
\def\labelenumi{\arabic{enumi}.}
\item
  当 \(0\leq l<1\) 时,\(\displaystyle \sum_{n=1}^\infty a_n\) 收敛
\item
  当 \(1 < l \leq +\infty\) 时,\(\displaystyle \sum_{n=1}^\infty a_n\)
  发散
\item
  当 \(l=1\) 时,判别法失效
\end{enumerate}
\end{theorem}

\begin{remark}
    当 \(a_n\) 中含有 \(n\) 次方时,可以考虑使用根值法
\end{remark}

\begin{remark}
    比值法和根值法实际上可看作是在将级数与等比级数作比较,均智能判断收敛速度不满与等比级数的级数.当所求级数存在时,可称级数为\textbf{拟等比级数}
\end{remark}

\begin{remark}
    根值法优于比值法
    \begin{itemize}
\item
  \(\displaystyle\lim_{n\to\infty}\frac{a_{n+1}}{a_n} =  l \Rightarrow \displaystyle \sqrt[n]{a_n} = l\)
\item
  \(\displaystyle \sqrt[n]{a_n} =l \nRightarrow \displaystyle\lim_{n\to\infty}\frac{a_{n+1}}{a_n} =  l\)
\end{itemize}
\end{remark}

\subsubsection{积分判别法}

\section{任意项级数的敛散性}

\subsection{错级数敛散性的判别法}

\subsubsection{交错级数}

\subsubsection{Leibniz 判别法}

\subsection{Abel 判别法和 Dirichlet 判别法}

\subsection{绝对收敛与条件收敛}

\section{函数项级数}

\section{幂级数}

\subsection{幂级数及其收敛半径}

\subsubsection{Abel 定理}

\section{幂级数收敛半径的求法}

\subsection{系数模比值法}

\subsection{系数模根值法}

\section{幂级数的性质}

\subsection{幂级数的分析性质}

\subsection{Taylor 级数}

\subsection{常用的初等函数的幂级数展开式}

\subsection{正弦级数和余弦级数}

\subsection{周期为 $2l$ 的 Fourier 级数}

\begin{assumption}
    这是一个假设环境。
\end{assumption}

\begin{axiom}
    这是一个公理环境。
\end{axiom}

\begin{conjecture}
    这是一个猜想环境。
\end{conjecture}

\nocite{*}
\bibliography{ref}
\end{document}