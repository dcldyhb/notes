\documentclass[lang = zh , final , oneside , openany , titlepage , zihao = -4 , linespread = 1.3 , baselineskip = false , cjk-font = windows , text-font = newtx , math-font = newtx]{sjtureport}

\usepackage{amsmath}
\usepackage{amsthm}
\usepackage[hidelinks]{hyperref}%目录超链接并隐藏
\usepackage{graphicx}%图片路径支持
\usepackage{gbt7714}
\bibliographystyle{gbt7714-numerical}
\usepackage{booktabs} % 表格支持
\usepackage{longtable} % 表格支持
%\usepackage{subcaption}%小标题支持

%\sjtusetup
%{
%  info =
%  {
%    zh/title = {上海交通大学学位论文模板示例文档},
%    en/title = {A Sample Document for SJTU Thesis Template},
%    zh/author = {某某},
%    en/author = {Mo Mo},
%  },
%
%  style =
%  {
%    float-num-sep = {-},
%  },
%
%  name =
%  {
%    achv = {攻读学位期间完成的论文},
%  },
%}

\title{高数2笔记}
\author{dcldyhb}
\subject{高等数学2}
\keywords{上海交大, 饮水思源, 爱国荣校}

\begin{document}

\maketitle

\setcounter{page}{1}  % 将页码设置为1
\pagestyle{plain}     % 设置为普通页码样式
\tableofcontents

%\begin{abstract}
%本模板是上海交通大学本科生课程论文、学位论文、学术报告等文档的LaTeX模板,旨在帮助学生快速上手LaTeX排版。
%\end{abstract}

\newpage
\setcounter{page}{1}  % 将页码设置为1
\pagestyle{plain}     % 设置为普通页码样式
\chapter{重积分}
\section{重积分的概念和性质}
\begin{definition}
    设 $D$ 是平面上的有界闭区域,$f(x,y)$ 为 $D$ 上的有界函数,$I$ 为实数.若对 $D$ 的任意分割 $\Delta D_1 , \Delta D_2 , \cdots , \Delta D_n$ ,任取 $(\xi_i , \eta_i)\in\Delta D_i(i=1,\ldots,n)$,作和 $\displaystyle\sum_{i=1}^nf(\xi_i,\eta_i)\Delta \sigma_i$ ($\Delta \sigma_i$ 为 $D_i$ 的面积),总有

    $$
    \lim_{\lambda\to 0}\sum_{i=1}^nf(\xi_i,\eta_i)\Delta \sigma_i = I
    $$

    其中 $\displaystyle\lambda = \max_{1\leq i\leq d} \{d_i\}$,$d_i$ 是小区域 $\Delta D_i$ 的直径,则称函数 $f(x,y)$ 在 $D$ 上\textbf{可积},记为 $f\in R(D)$;极限值 $I$ 称为 $f(x,y)$ 在 $D$ 上的\textbf{二重积分},记作

    $$
    \iint_D f(x,y)\,\mathrm{d}\sigma.
    $$
\end{definition}

\begin{enumerate}
    \item $\displaystyle \iint$ 积分号
    \item $D$ 积分区域
    \item $f(x,y)$ 被积函数
    \item $\mathrm{d}\sigma$ 面积元素(微元)
    \item 二重积分的几何意义
        \begin{enumerate}
            \item 当被积函数大于 $0$ 时,二重积分是柱体体积
            \item 当被积函数小于 $0$ 时,二重积分是柱体体积的负值
            \item 一般的,为曲顶柱体体积的代数和
        \end{enumerate}
    \item 可积的充分条件
        \begin{enumerate}
            \item 若函数 $f(x,y)$ 在区域 $D$ 上连续,则 $f(x,y)\in D$
        \end{enumerate}
    \item $f(x,y)$ 在 $D$ 上的可积性及积分值与其在 $D$ 内\textbf{有限条光滑曲线}上的定义无关
\end{enumerate}

\section{二重积分的性质}

\begin{enumerate}
    \item $\displaystyle\iint_D\,\mathrm{d}\sigma = \iint_D 1\,\mathrm{d}\sigma = A_D \quad\left(\text{D 的面积}\right)$.
    \item **线性性:** 设 $f,g\in R(D)$,$\alpha , \beta$,是任意常数,则 $\alpha f +\beta g \in R(D)$,且

    $$
    \iint_D (\alpha f +\beta g)\,\mathrm{d}\sigma = \alpha \iint_D f\,\mathrm{d}\sigma + \beta \iint_D g\,\mathrm{d}\sigma
    $$
\end{enumerate}

\begin{assumption}
    这是一个假设环境。
\end{assumption}

\begin{axiom}
    这是一个公理环境。
\end{axiom}

\begin{conjecture}
    这是一个猜想环境。
\end{conjecture}

\nocite{*}
\bibliography{ref}
\end{document}