\documentclass[lang = zh , final , oneside , openany , titlepage , zihao = -4 , linespread = 1.3 , baselineskip = false , cjk-font = windows , text-font = newtx , math-font = newtx]{sjtureport}

\usepackage{amsmath}
\usepackage{amsthm}
\usepackage[hidelinks]{hyperref}%目录超链接并隐藏
\usepackage{graphicx}%图片路径支持
\usepackage{gbt7714}
\bibliographystyle{gbt7714-numerical}
\usepackage{booktabs} % 表格支持
\usepackage{longtable} % 表格支持
%\usepackage{subcaption}%小标题支持

%\sjtusetup
%{
%  info =
%  {
%    zh/title = {上海交通大学学位论文模板示例文档},
%    en/title = {A Sample Document for SJTU Thesis Template},
%    zh/author = {某某},
%    en/author = {Mo Mo},
%  },
%
%  style =
%  {
%    float-num-sep = {-},
%  },
%
%  name =
%  {
%    achv = {攻读学位期间完成的论文},
%  },
%}

\title{上海交通大学报告模板示例文档}
\author{某某}
\subject{XX期末课程论文}
\keywords{上海交大, 饮水思源, 爱国荣校}

\begin{document}

\maketitle

\setcounter{page}{1}  % 将页码设置为1
\pagestyle{plain}     % 设置为普通页码样式
\tableofcontents

以下是将上述内容转换为 LaTeX 格式的源码,从 `\chapter` 开始:

```latex
\documentclass{book}
\usepackage{amsmath}
\usepackage{amsfonts}
\usepackage{amssymb}
\usepackage{mathrsfs}
\usepackage{graphicx}
\usepackage{hyperref}
\usepackage{booktabs}
\usepackage{multirow}
\usepackage{caption}
\usepackage{subcaption}
\usepackage{float}
\usepackage{enumitem}
\usepackage{mathtools}
\usepackage{bm}

% 设置 MathJax (如果需要在网页上显示)
% 这里注释掉,因为 LaTeX 编译不需要 MathJax
% \usepackage{mathjax}
% \mathjax{https://cdn.mathjax.org/mathjax/latest/MathJax.js?config=TeX-AMS-MML_HTMLorMML}

\begin{document}

\chapter{高等数学 2 笔记}

\section{重积分}

\subsection{重积分的概念和性质}

\subsubsection{二重积分的概念}

\textbf{定义:}

设 \(D\) 是平面上的有界闭区域,\(f(x,y)\) 为 \(D\) 上的有界函数,\(I\) 为实数。若对 \(D\) 的任意分割 \(\Delta D_1, \Delta D_2, \cdots, \Delta D_n\) ,任取 \((\xi_i, \eta_i)\in\Delta D_i(i = 1,\ldots,n)\),作和 \(\sum_{i = 1}^n f(\xi_i,\eta_i)\Delta \sigma_i\) (\(\Delta \sigma_i\) 为 \(D_i\) 的面积),总有

\[
\lim_{\lambda\to 0}\sum_{i = 1}^n f(\xi_i,\eta_i)\Delta \sigma_i = I
\]

其中 \(\lambda = \max_{1\leq i\leq n} \{d_i\}\),\(d_i\) 是小区域 \(\Delta D_i\) 的直径,则称函数 \(f(x,y)\) 在 \(D\) 上\textbf{可积},记为 \(f\in R(D)\);极限值 \(I\) 称为 \(f(x,y)\) 在 \(D\) 上的\textbf{二重积分},记作

\[
\iint_D f(x,y)\,\mathrm{d}\sigma.
\]

\begin{itemize}
    \item \(\iint\) 积分号
    \item \(D\) 积分区域
    \item \(f(x,y)\) 被积函数
    \item \(\mathrm{d}\sigma\) 面积元素(微元)
\end{itemize}

- 二重积分的几何意义
  - 当被积函数大于 \(0\) 时,二重积分是柱体体积
  - 当被积函数小于 \(0\) 时,二重积分是柱体体积的负值
  - 一般的,为曲顶柱体体积的代数和

- 可积的充分条件
  - \textbf{定理:} 若函数 \(f(x,y)\) 在区域 \(D\) 上连续,则 \(f(x,y)\in D\)。

- \(f(x,y)\) 在 \(D\) 上的可积性及积分值与其在 \(D\) 内\textbf{有限条光滑曲线}上的定义无关

\subsubsection{二重积分的性质}

\begin{enumerate}
    \item \(\iint_D\,\mathrm{d}\sigma = \iint_D 1\,\mathrm{d}\sigma = A_D \quad\left(\text{D 的面积}\right)\)。
    \item \textbf{线性性:} 设 \(f,g\in R(D)\),\(\alpha, \beta\) 是任意常数,则 \(\alpha f +\beta g \in R(D)\),且

    \[
    \iint_D (\alpha f +\beta g)\,\mathrm{d}\sigma = \alpha \iint_D f\,\mathrm{d}\sigma + \beta \iint_D g\,\mathrm{d}\sigma
    \]

    \item \textbf{区域可加性:} 若 \(f\in R(D)\) 且积分区域 \(D\) 分为内部不相交的子区域 \(D_1, D_2\),则

    \[
    \iint_D f(x,y) \,\mathrm{d}\sigma = \iint_{D_1} f(x,y)\,\mathrm{d}\sigma + \iint_{D_2} f(x,y)\,\mathrm{d}\sigma
    \]

    \item \textbf{保序性:} 若 \(f,g\in R(D)\),且当 \((x,y)\in D\) 时,\(f(x,y)\leq g(x,y)\),则

    \[
    \iint_D f(x,y)\,\mathrm{d}\sigma \leq \iint_D g(x,y)\,\mathrm{d}\sigma
    \]

    - \textbf{推论 1:} 若 \(f\in R(D)\),则 \(\vert f(x,y) \vert \in R(D)\),且

    \[
    \left\vert \iint_D f(x,y)\,\mathrm{d}\sigma \right\vert \leq \iint_D \vert f(x,y) \vert \,\mathrm{d}\sigma
    \]

    - \textbf{推论 2:} 若 \(f\in R(D)\),且当 \((x,y)\in D\) 时, \(m\leq f(x,y) \leq M\),则

    \[
    mA_D \leq \iint_D f(x,y)\,\mathrm{d}\sigma \leq MA_D
    \]

    - \textbf{推论 3:} 若 \(f\in R(D)\),且当 \((x,y)\in D\) 时,\(f(x,y) \geq 0\),则

    \[
    \iint_D f(x,y)\,\mathrm{d}\sigma \geq 0
    \]

    \item \textbf{积分中值定理:} 若 \(f(x,y)\in C(D)\) ,\(g(x,y)\in R(D)\),且 \(g\) 在 \(D\) 上不变号,则 \(\exists \xi, \eta \in D\),使得

    \[
    \iint_D f(x,y)g(x,y)\,\mathrm{d}\sigma = f(\xi, \eta )\iint_D g(x,y)\,\mathrm{d}\sigma
    \]

    - \textbf{推论:} 若 \(f(x,y)\in C(D)\),则存在 \((\xi, \eta )\in D\),使得

    \[
    \iint_D f(x,y)\,\mathrm{d}\sigma = f(\xi, \eta )A_D
    \]

    称 \(f(\xi,\eta) = \frac{\iint_D f\,\mathrm{d}\sigma}{A_D}\) 为函数 \(f(x,y)\) 在有界闭区域 \(D\) 上的\textbf{平均值}
\end{enumerate}

\subsection{二重积分的计算}

\subsubsection{直角坐标系下的计算}

当二重积分存在时,可利用平行于坐标轴的直线来划分积分区域 \(D\),此时,面积元素

\[
\mathrm{d}\sigma = \mathrm{d}x\mathrm{d}y
\]

故二重积分在直角坐标系下可表示为

\[
\iint_D f(x,y)\,\mathrm{d}\sigma = \iint_D f(x,y)\,\mathrm{d}x\mathrm{d}y
\]

\paragraph{\(x\) 型正则区域}

\[
D = \left\{(x,y)\mid \varphi_1(x)\leq y\leq \varphi_2(x),a\leq x\leq b\right\}
\]

化为先 \(y\) 后 \(x\) 的二次积分

\[
\begin{aligned}
    \iint_D f(x,y)\,\mathrm{d}x\mathrm{d}y &= \int_a^b\left[\int_{\varphi_1(x)}^{\varphi_2(x)} f(x,y)\,\mathrm{d}y\right]\mathrm{d}x \\
    &\equiv \int_a^b f(x,y)\,\mathrm{d}x\mathrm{d}y
\end{aligned}
\]

\paragraph{\(y\) 型正则区域}

\[
D = \left\{(x,y)\mid \varphi_1(y)\leq x\leq \varphi_2(y),c\leq y\leq d\right\}
\]

化为先 \(x\) 后 \(y\) 的二次积分

\[
\begin{aligned}
    \iint_D f(x,y)\,\mathrm{d}x\mathrm{d}y &= \int_c^d\left[\int_{\varphi_1(y)}^{\varphi_2(y)} f(x,y)\,\mathrm{d}x\right]\mathrm{d}y \\
    &\equiv \int_c^d f(x,y)\,\mathrm{d}x\mathrm{d}y
\end{aligned}
\]

\paragraph{一般区域的二重积分}

分割成若干个正则子区域,利用积分区域可加性,分别在子区域上积分后求和

\begin{quote}
直角坐标计算二重积分的步骤

1. \textbf{画出积分区域} \(D\) 的草图,并\textbf{确定类型}
2. 按照所确定的类型\textbf{表示区域} \(D\)
3. \textbf{化二重积分为二次积分}(注意上下限)
4. \textbf{计算}二重积分
\end{quote}

\subsubsection{极坐标系下的计算}

当积分区域的边界曲线或被积函数用极坐标表示较为简单时,可用极坐标来计算二重积分。

面积元素 \(\Delta \sigma\) 在极坐标下为

\[
\boxed{\Delta\sigma = r\mathrm{d}r\mathrm{d}\theta}
\]

从直角坐标到极坐标时的二重积分变换公式为

\[
\iint_D f(x,y)\,\mathrm{d}\sigma = \iint_D f(r\cos\theta,r\sin\theta)r\,\mathrm{d}r\mathrm{d}\theta
\]

若积分区域 \(D\) 可表示为 \(\left\{(r,\theta)\mid r_1(\theta)\leq r \leq r_2(\theta), \alpha \leq \theta \leq \beta \right\}\),则

\[
\iint_D f(x,y)\,\mathrm{d}\sigma = \int_\alpha^\beta\, \mathrm{d}\theta\int_{r_1(\theta)}^{r_2(\theta)} f(r\cos\theta,r\sin\theta)r\,\mathrm{d}r
\]

\subsubsection{二重积分的变量代换}

\textbf{定理:}

设变换 \(T :\begin{cases}x = x(u,v)\\y = y(u,v)\end{cases}\) 有连续偏导数,且满足 \(J = \frac{\partial (x,y)}{\partial (u,v)}\coloneqq \begin{vmatrix}x_u & x_v\\ y_u & y_v\end{vmatrix} \left(\text{Jacobi 行列式}\right)\neq 0\),而 \(f(x,y)\in C(D)\),则

\[
\iint_D f(x,y)\,\mathrm{d}\sigma = \iint_{D^*} f(x(u,v),y(u,v))\left\vert J \right\vert \,\mathrm{d}u\mathrm{d}v
\]

- 在定理条件下,变换 \(T\) 一定存在逆变换 \(T^{-1}:\begin{cases}u = u(x,y)\\v = v(x,y)\end{cases}\),且 \(\frac{\partial (u,v)}{\partial (x,y)}\cdot \frac{\partial (x,y)}{\partial (u,v)} = 1\)

  有时,借助此式求 \(J\) 较为简单

- 当 Jacobi 行列式仅在区域 \(D^*\) 内个别点上或个别曲线上为 \(0\) 时,定理结论仍成立
- 在广义极坐标 \(\begin{cases}x = ar\cos\theta\\y = br\sin\theta\end{cases}\) 下,\(J = abr\)

\subsection{三重积分}

\subsubsection{三重积分的定义}

\textbf{定义:}

设 \(\Omega\) 是 \(\mathbb{R}^3\) 中的有界闭区域,三元函数 \(f(x,y,z)\) 在 \(\Omega\) 上有界,\(I\) 为实数。 若将 \(\Omega\) 任意分割成 \(n\) 个小区域 \(\Delta \Omega_1,\Delta\Omega_2,\ldots,\Delta\Omega_n\),任取 \((\varepsilon_i,\eta_i,\xi_i )\in\Delta\Omega_i\left(i = 1,2,\ldots,n\right)\),作和 \(\sum_{i = 1}^n f(\varepsilon_i,\eta_i,\xi_i )\Delta V_i\),(\(\Delta V_i\) 是 \(\Delta\Omega_i\) 的体积),总有

\[
\lim_{\lambda\to 0}\sum_{i = 1}^n f(\varepsilon_i,\eta_i,\xi_i )\Delta V_i = I
\]

其中 \(\lambda = \max_{1\leq i\leq n}\{d_i\}\),\(d_i\) 是 \(\Delta\Omega_i\) 的直径,则称函数 \(f(x,y,z)\) 在 \(\Omega\) 上\textbf{可积},记为 \(f\in R(\Omega)\); \(I\) 称为 \(f(x,y,z)\) 在 \(\Omega\) 上的\textbf{三重积分},记作

\[
\iiint_\Omega f(x,y,z)\,\mathrm{d}V
\]

其中 \(V_\Omega\) 是区域 \(\Omega\) 的体积

- 若 \(f(x,y,z)\) 表示占有三维空间区域 \(\Omega\) 的物体的体密度函数,则 \(\iiint_\Omega f(x,y,z)\,\mathrm{d}V\) 给出了物体的\textbf{质量}
- 类似二重积分,三重积分具有线性性,区域可加性,保序性以及推论和积分中值定理,并且有 \(\iiint_\Omega \,\mathrm{d}V = V_\Omega\)

\subsubsection{直角坐标系下的计算}

直角坐标系下,由于 \(\mathrm{d}V = \mathrm{d}x\mathrm{d}y\mathrm{d}z\),故

\[
\iiint_\Omega f(x,y,z)\,\mathrm{d}V = \iiint_\Omega f(x,y,z)\,\mathrm{d}x\mathrm{d}y\mathrm{d}z
\]

\paragraph{柱线法(坐标面投影法)}

设 \(\Omega\) 是以曲面 \(z = z_1(x,y)\) 为底,曲面 \(z = z_2(x,y)\),而侧面是母线平行于 \(z\) 轴的柱体所围成的区域

设 \(\Omega\) 在 \(xOy\) 面上的投影区域为 \(D_1\) ,则 \(\Omega\) 可表示为

\[
\Omega = \left\{(x,y,z)\mid (x,y)\in D_1, z_1(x,y)\leq z\leq z_2(x,y),(x,y)\in D\right\}
\]

则物体总质量为

\[
\iint_D\left(\int_{z_1(x,y)}^{z_2(x,y)}f(x,y,z)\,\mathrm{d}z\right) \,\mathrm{d}x\mathrm{d}y
\]

故有

\[
\iiint_\Omega f(x,y,z)\,\mathrm{d}V = \iint_{D_1} \,\mathrm{d}x\mathrm{d}y\int_{z_1(x,y)}^{z_2(x,y)} f(x,y,z)\,\mathrm{d}z
\]

\paragraph{截面法(坐标轴投影法)}

设区域 \(\Omega\) 在 \(z\) 轴上的投影区间为 \([h_1,h_2]\),即 \(\Omega\) 介于平面 \(z = h_1\) 和 \(z = h_2\) 之间,过 \(z\) 处且垂直于 \(z\) 轴的平面截 \(\Omega\) 得截面区域 \(D_z\),则 \(\Omega\) 可表示为

\[
\Omega = \left\{(x,y,z)\mid h_1\leq z\leq h_2, (x,y)\in D_z\right\}
\]

物体总质量为

\[
\int_{h_1}^{h_2}\left(\iint_{D_z} f(x,y,z)\,\mathrm{d}x\mathrm{d}y\right) \,\mathrm{d}z
\]

故有

\[
\iiint_\Omega f(x,y,z)\,\mathrm{d}V = \int_{h_1}^{h_2}\,\mathrm{d}z\iint_{D_z} f(x,y,z)\,\mathrm{d}x\mathrm{d}y
\]

\subsubsection{三重积分的变量代换}

\textbf{定理:}

设变换 \(T:\begin{cases}x = x(u,v,w)\\y = y(u,v,w)\\z = z(u,v,w)\end{cases}\) 有连续偏导数,且满足 \(J = \frac{\partial (x,y,z)}{\partial (u,v,w)}\neq 0\),而 \(f(x,y,z)\in C(\Omega)\),则

\[
\iiint_\Omega f(x,y,z)\,\mathrm{d}V = \iiint_{\Omega^*} f(x(u,v,w),y(u,v,w),z(u,v,w))\left\vert J \right\vert \,\mathrm{d}u\mathrm{d}v\mathrm{d}w
\]

\paragraph{柱面坐标系下的计算}

柱面坐标系,实际上就是将 \(x,y\) 坐标转换为极坐标

\[
\begin{cases}
x = r\cos\theta\\
y = r\sin\theta\\
z = z
\end{cases}
\]

其 Jacobi 行列式为

\[
J = \frac{\partial (x,y,z)}{\partial (r,\theta,z)} = \begin{vmatrix}
\cos\theta & -r\sin\theta & 0\\
\sin\theta & r\cos\theta & 0\\
0 & 0 & 1 
\end{vmatrix} = r
\]

则柱面积分积分公式为

\[
\iiint_\Omega f(x,y,z)\,\mathrm{d}V = \iiint_{\Omega^*} f(r\cos\theta,r\sin\theta,z)r\,\mathrm{d}r\mathrm{d}\theta\mathrm{d}z
\]

\paragraph{球面坐标系下的计算}

球面坐标系,实际上就是将 \(x,y,z\) 坐标转换为球坐标

\[
\begin{cases}
x = \rho\sin\varphi\cos\theta\\
y = \rho\sin\varphi\sin\theta\\
z = \rho\cos\varphi
\end{cases}
\]

其 Jacobi 行列式为

\[
J = \frac{\partial (x,y,z)}{\partial (\rho,\varphi,\theta)} = \begin{vmatrix}
\sin\varphi\cos\theta & \rho\cos\varphi\cos\theta & -\rho\sin\varphi\sin\theta\\
\sin\varphi\sin\theta & \rho\cos\varphi\sin\theta & \rho\sin\varphi\cos\theta\\
\cos\varphi & -\rho\sin\varphi & 0
\end{vmatrix} = \rho^2\sin\varphi
\]

则球面积分积分公式为

\[
\iiint_\Omega f(x,y,z)\,\mathrm{d}V = \iiint_{\Omega^*} f(\rho\sin\varphi\cos\theta,\rho\sin\varphi\sin\theta,\rho\cos\varphi)\rho^2\sin\varphi\,\mathrm{d}\rho\mathrm{d}\varphi\mathrm{d}\theta
\]

\subsection{重积分的应用}

\subsubsection{重积分的几何应用}

\paragraph{平面图形的面积}

\[
A(D) = \iint_D \,\mathrm{d}\sigma = \iint_D \,\mathrm{d}x\mathrm{d}y
\]

\paragraph{立体的体积}

\[
V(\Omega) = \iiint_\Omega \,\mathrm{d}V = \iiint_\Omega \,\mathrm{d}x\mathrm{d}y\mathrm{d}z
\]

\paragraph{曲面的面积}

设空间曲面 \(S:z = f(x,y),(x,y)\in D\)。

则曲面 \(S\) 的面积为

\[
A(S) = \iint_D \sqrt{1 + z_x^2 + z_y^2}\,\mathrm{d}x\mathrm{d}y
\]

\subsubsection{重积分的物理应用}

\paragraph{质心}

体密度为 \(\rho(x, y)\) 的物体占据空间 \(\Omega\),其质心坐标为

\[
\begin{cases}
  \bar{x} = \frac{\iiint_\Omega x\rho(x,y,z)\,\mathrm{d}V}{\iiint_\Omega\rho(x,y,z)\,\mathrm{d}V} \\
  \bar{y} = \frac{\iiint_\Omega y\rho(x,y,z)\,\mathrm{d}V}{\iiint_\Omega\rho(x,y,z)\,\mathrm{d}V}\\
  \bar{z} = \frac{\iiint_\Omega z\rho(x,y,z)\,\mathrm{d}V}{\iiint_\Omega\rho(x,y,z)\,\mathrm{d}V}
\end{cases}
\]

\paragraph{转动惯量}

设物体的密度为 \(\rho(x,y,z)\),则物体分别关于 \(x\),\(y\),\(z\) 轴的转动惯量为

\[
\begin{cases}
  I_x = \iiint_\Omega \rho(x,y,z)(y^2 + z^2)\,\mathrm{d}V\\
  I_y = \iiint_\Omega \rho(x,y,z)(x^2 + z^2)\,\mathrm{d}V\\
  I_z = \iiint_\Omega \rho(x,y,z)(x^2 + y^2)\,\mathrm{d}V
\end{cases}
\]

\paragraph{引力}

\[
\begin{aligned}
  \mathrm{d}\vec{F} &= G\frac{m_0\mathrm{d}m}{r^3}\vec{r}\\
  &= G\frac{m_0\rho(x,y,z)\mathrm{d}V}{r^3}\cdot\left(x - x_0,y - y_0,z - z_0\right)\\
  &= \left(\mathrm{d}F_x,\mathrm{d}F_y,\mathrm{d}F_z\right)
\end{aligned}
\]

\section{曲线积分和曲面积分}

\subsection{第一类曲线积分和曲面积分}

\subsubsection{第一类曲线积分的概念}

\textbf{定义:}

设 \(C\) 是 \(xOy\) 面上的一条光滑曲线弧,函数 \(f(x,y)\) 是定义在 \(C\) 上的有界函数,在 \(C\) 上任意插入分点 \(A = A_0,A_1,\ldots,A_{n - 1},A_n = B\),将其分成 \(n\) 个小弧段,记第 \(i\) 个小弧段的弧长为 \(\Delta s_i\),在第 \(i\) 个小段上任取点 \((\epsilon_i,\eta_i)\),和式 \(\sum_{i = 1}^{n}f(\epsilon_i,\eta_i)\Delta s_i\),当 \(\lambda=\max_{1\leq i\leq n}\{\Delta s_i\}\to 0\) 时,有确定的极限值 \(I\),即

\[
\lim_{\lambda\to 0}\sum_{i = 1}^n f(\epsilon_i,\eta_i)\Delta s_i = I
\]

则称函数 \(f(x,y)\) 在曲线 \(C\) 上\textbf{可积},并将此极限值 \(I\) 称为函数 \(f(x,y)\) 在曲线 \(C\) 上的\textbf{第一类曲线积分},记作 \(\int_Cf(x,y)\,\mathrm{d}s\),即

\[
I = \int_C f(x,y)\,\mathrm{d}s = \lim_{\lambda\to 0}\sum_{i = 1}^n f(\epsilon_i,\eta_i)\Delta s_i
\]

- 第一类曲线积分的几何含义为柱面的面积
- \(\int_C\,\mathrm{d}s = \int_C1\,\mathrm{d}s = s_C\)
- 若 \(C\) 是封闭曲线,即 \(C\) 的二端点重合,则记第一类曲线积分为 \(\oint_Cf(x,y)\,\mathrm{d}s\)

\subsubsection{第一类曲线积分的性质}

\paragraph{与曲线方向无关}

若曲线 \(C\) 的端点为 \(A\) 和 \(B\),\(f(x,y)\) 在曲线 \(C\) 上可积,则

\[
\int_{\widehat{AB}} f(x,y)\,\mathrm{d}s=\int_{\widehat{BA}} f(x,y)\,\mathrm{d}s 
\]

\paragraph{线性性}

若 \(f,g\) 在曲线 \(C\) 上可积,\(\alpha, \beta\) 是任意常数,则 \(\alpha f + \beta g\) 在曲线 \(C\) 上可积,且

\[
\int_C (\alpha f + \beta g)\,\mathrm{d}s = \alpha \int_C f(x,y)\,\mathrm{d}s + \beta \int_C g(x,y)\,\mathrm{d}s
\]

\paragraph{路径可加性}

若曲线 \(C\) 由两段光滑曲线 \(C_1\) 和 \(C_2\) 首尾连接而成,则 \(f(x,y)\) 在曲线 \(C\) 上可积,等价于 \(f(x,y)\) 在曲线 \(C_1\) 和 \(C_2\) 上均可积,且

\[
\int_C f(x,y)\,\mathrm{d}s = \int_{C_1} f(x,y)\,\mathrm{d}s + \int_{C_2} f(x,y)\,\mathrm{d}s
\]

\paragraph{中值定理}

设函数 \(f\) 在光滑曲线 \(C\) 上连续,则 \(\exists (\epsilon,\eta)\in C\),使得

\[
\int_C f(x,y)\,\mathrm{d}s = f(\epsilon,\eta)\cdot s_C
\]

其中 \(s_C\) 是曲线段 \(C\) 的长度

\subsubsection{第一类曲线积分的计算}

设函数 \(f(x,y)\) 在光滑曲线 \(C\) 上连续,\(C\) 的参数方程为 \(\begin{cases}x = x(t)\\y = y(t)\end{cases}\),\(t\in [a,b]\),满足 \(x'(t)\),\(y'(t)\) 连续,且 \(x'(t)^2 + y'(t)^2 \neq 0\),则

\[
\int_C f(x,y)\,\mathrm{d}s = \int_a^b f(x(t),y(t))\sqrt{x'(t)^2 + y'(t)^2}\,\mathrm{d}t
\]

- 右端积分限应 \(a < b\)
- 当曲线 \(C\) 形式为 \(y = y(x)\),\(x\in [a,b]\) 时,则 \(\int_Cf(x,y)\,\mathrm{d}s =\int_a^bf(x,y(x))\sqrt{1 + y'^2(x)}\,\mathrm{d}x\)
- 当曲线 \(C\) 为极坐标 \(r = r(\theta)\),\(\theta\in [\alpha,\beta]\) 时,则 \(\int_Cf(x,y)\,\mathrm{d}s = \int_\alpha^\beta f(r(\theta)\cos\theta,r(\theta)\sin\theta)\sqrt{r(\theta)^2 + r'^2(\theta)}\,\mathrm{d}\theta\)

\subsection{第二类曲线积分与曲面积分}

\subsubsection{第二类曲线积分的概念}

\textbf{定义:}

设 \(C\) 为平面光滑定向曲线(\(A\rightarrow B\)),且向量值函数 \(\vec{F}(x,y) = P(x,y)\vec{i}+Q(x,y)\vec{j}\) 在 \(C\) 上有界,\(\vec{e}_\tau\) 为 \(C\) 上点 \((x,y)\) 处于定向一致的单位切向量,若

\[
\int_C \vec{F}(x,y) \cdot \vec{e}_\tau \,\mathrm{d}s
\]

存在,则称为\textbf{向量值函数 \(\vec{F}\) 在定向曲线 \(C\) 上的曲线积分或第二类曲线积分}

若 \(\vec{e}_\tau(x,y) = (\cos\alpha,\cos\beta)\),则

\[
\begin{aligned}
  \int_C \vec{F}(x,y) \cdot \vec{e}_\tau \,\mathrm{d}s &= \int_C P(x,y)\cos\alpha + Q(x,y)\cos\beta \,\mathrm{d}s \\
  &= \int_C P(x,y)\cos\alpha \,\mathrm{d}s + \int_C Q(x,y)\cos\beta \,\mathrm{d}s \\
  &= \int_C P(x,y)\,\mathrm{d}x +  Q(x,y)\,\mathrm{d}y
\end{aligned}
\]

这是对坐标的曲线积分

记 \(\vec{r} = (x,y)\),则 \(\mathrm{d}\vec{r} = \vec{e}_\tau \,\mathrm{d}s\) 称为\textbf{定向弧微分}

从而有向量形式的第二类曲线积分

\[
\int_C \vec{F}(x,y) \cdot \mathrm{d}\vec{r} = \int_C \vec{F}\cdot \mathrm{d}\vec{r}
\]

\paragraph{第二类曲线积分的性质}

第二类曲线积分与\textbf{曲线方向有关},即

\[
\int_{\widehat{AB}} \vec{F}(x,y) \cdot \mathrm{d}\vec{r} = -\int_{\widehat{BA}} \vec{F}(x,y) \cdot \mathrm{d}\vec{r}
\]

此外线性性与对定向积分路径的可加性等仍然成立

\paragraph{第二类曲线积分的计算}

若曲线 \(C\) 为 \(\begin{cases}x = x(t)\\y = y(t)\end{cases}\),\(t:\alpha\rightarrow \beta\)

起点 \(A\) 对应 \(\alpha\),终点 \(B\) 对应 \(\beta\)

考察 \(\int_C P\,\mathrm{d}x + Q\,\mathrm{d}y = \int_C \vec{F}\cdot\vec{e}_\tau \,\mathrm{d}s\),沿曲线 \(C\) 有 \(\vec{F} = \left(P\left(x(t),y(t)\right),Q\left(x(t),y(t)\right)\right)\),则

\[
\int_C P\,\mathrm{d}x + Q\,\mathrm{d}y = \int_\alpha^\beta P\left(x(t),y(t)\right)\,\mathrm{d}x(t) + Q\left(x(t),y(t)\right)\,\mathrm{d}y(t)
\]

\subsubsection{第二类曲面积分的概念}

\paragraph{双侧曲面}

\textbf{定义:}

若点 \(P\) 沿曲面 \(S\) 上任何不越过曲面边界的连续闭曲线移动后回到起始位置时,法向量 \(\vec{n}\) 保持原来的指向,则称 \(S\) 为\textbf{双侧曲面}

典型的,Mobius 面不是双侧曲面

选定双侧曲面 \(S\) 一侧为正向,称为\textbf{正侧},记为 \(S^+\) ,其相反侧记作 \(S^-\)

\paragraph{双侧曲面定侧}

若 \(S:z = z(x,y)\),\((x,y)\in D_{xy}\),\(\vec{n}_0=\left(\cos\alpha,\cos\beta,\cos\gamma\right)=\pm\frac{\left(-z_x,-z_y,1\right)}{\sqrt{1 + z_x^2 + z_y^2}}\)

若选取 \(\vec{n}_0=\left(\cos\alpha,\cos\beta,\cos\gamma\right)=\frac{\left(-z_x,-z_y,1\right)}{\sqrt{1 + z_x^2 + z_y^2}}\),则说明 \(\cos\gamma > 0\),选取了曲面的上侧

一般的

\[
\begin{cases}
  \cos\alpha > 0 \Leftrightarrow \text{前侧},\cos\alpha < 0 \Leftrightarrow \text{后侧}\\
  \cos\beta > 0 \Leftrightarrow \text{右侧},\cos\beta < 0 \Leftrightarrow \text{左侧}\\
  \cos\gamma > 0 \Leftrightarrow \text{上侧},\cos\gamma < 0 \Leftrightarrow \text{下侧}
\end{cases}
\]

习惯上选取曲面片的上侧为 \(S^+\);对于封闭曲面,选取外侧为 \(S^+\)

对于向量值函数 \(\vec{F} = (P,Q,R)\)

\[
\iint_S \vec{F}\cdot\,\mathrm{d}\vec{S} = \iint_S P\,\mathrm{d}x\mathrm{d}y + Q\,\mathrm{d}y\mathrm{d}z + R\,\mathrm{d}z\mathrm{d}x
\]

\paragraph{第二类曲面积分的性质}

第二类曲面积分与在曲面的哪一侧积分有关

\[
\iint_{S^+}P\,\mathrm{d}x\mathrm{d}y + Q\,\mathrm{d}y\mathrm{d}z + R\,\mathrm{d}z\mathrm{d}x = -\iint_{S^-}P\,\mathrm{d}x\mathrm{d}y + Q\,\mathrm{d}y\mathrm{d}z + R\,\mathrm{d}z\mathrm{d}x
\]

此外第二类曲面积分也具有线性性和可加性等性质

\subsubsection{第二类曲面积分的计算}

\paragraph{合一投影法}

\[
\iint_{S^+}P\,\mathrm{d}x\mathrm{d}y + Q\,\mathrm{d}y\mathrm{d}z + R\,\mathrm{d}z\mathrm{d}x = \iint_{D_{xy}} \left(-Pz_x - Qz_y + R\right)\,\mathrm{d}x\mathrm{d}y
\]

\paragraph{分面投影法}

分 \(P\,\mathrm{d}x\mathrm{d}y\),\(Q\,\mathrm{d}y\mathrm{d}z\),\(R\,\mathrm{d}z\mathrm{d}x\) 三个部分进行积分

常在部分曲面垂直坐标轴时进行

\paragraph{公式法}

常用于参数方程确定的曲面

设 \(S:\vec{r} = \left(x(u,v),y(u,v),z(u,v)\right)\),其中 \((u,v)\in D_{uv}\),则

\[
\iint_{S^+}\vec{F}\cdot\,\mathrm{d}\vec{S} = \iint_{D_{uv}}\vec{F}\cdot\left(\vec{r}_u\times\vec{r}_v\right)\,\mathrm{d}u\mathrm{d}v
\]

\subsection{Green 公式及其应用}

\subsubsection{Green 公式}

\paragraph{连通区域及其边界方向}

设 \(D\) 为平面区域,若 \(D\) 内的任意一条闭曲线所围的区域都落在 \(D\) 内,则称 \(D\) 是单连通的,否则称 \(D\) 为复连通的

当点沿区域边界朝一个方向前进时,区域总在它的左侧,则将此方向规定为边界曲线 \(C\) 的正向,记为 \(C^+\),与 \(C^+\) 相反方向为 \(C^−\)

\paragraph{Green 公式}

\textbf{定理:}

设有界闭区域 \(D\) 由分段光滑曲线 \(C\) 围成,函数 \(P(x, y)\),\(Q(x, y)\) 在 \(D\) 上有一阶连续偏导数,则

\[
\oint_{C^+} P\,\mathrm{d}x + Q\,\mathrm{d}y = \iint_D \left(\frac{\partial Q}{\partial x} - \frac{\partial P}{\partial y}\right)\,\mathrm{d}x\mathrm{d}y
\]

- 对于复连通区域 \(D\),Green 公式仍然成立,但需将 \(C\) 分成若干个单连通区域 \(D_i\),并对每个区域应用 Green 公式
- 公式也可以记为 \(\oint_{C^+} P\,\mathrm{d}x + Q\,\mathrm{d}y = \iint_D \begin{vmatrix}\frac{\partial}{\partial x}&\frac{\partial}{\partial y}\\ P&Q\end{vmatrix}\,\mathrm{d}x\mathrm{d}y\)

\paragraph{Green 公式的向量形式}

\[
\oint_{C^+} \vec{F}\cdot\,\mathrm{d}\vec{r} = \iint_D \nabla\times\vec{F}\,\mathrm{d}x\mathrm{d}y
\]

\subsubsection{曲线积分与路径无关的条件}

\textbf{定义:}

设 \(P(x,y)\),\(Q(x,y)\) 在区域 \(D\) 内连续,若对 \(D\) 内任意两点 \(A\),\(B\) 以及 \(D\) 内连接 \(A,B\) 的任意二分段光滑曲线 \(C_1\),\(C_2\),均有

\[
\int_{C_1} P\,\mathrm{d}x +Q\,\mathrm{d}y = \int_{C_2} P\,\mathrm{d}x +Q\,\mathrm{d}y
\]

则称曲线积分 \(\int_C P\,\mathrm{d}x +Q\,\mathrm{d}y\) 在 \(D\) 内\textbf{与路径无关}

\textbf{定理:}

设函数  \(P\),\(Q\) 在\textbf{单连通}区域 \(D\) 上有连续偏导数,则下述四命题等价

\begin{enumerate}
    \item 在 \(D\) 内的任一条分段光滑闭曲线 \(C\) 上,有 \(\int_C P\,\mathrm{d}x +Q\,\mathrm{d}y = 0\)
    \item 曲线积分 \(\int_C P\,\mathrm{d}x +Q\,\mathrm{d}y\) 在 \(D\) 内与路径无关
    \item 存在 \(D\) 上的可微函数 \(u(x,y)\) 使得 \(\mathrm{d}u = P\,\mathrm{d}x +Q\,\mathrm{d}y\),此时称 \(u(x,y)\) 为 \(P\,\mathrm{d}x +Q\,\mathrm{d}y\) 的一个\textbf{原函数}
    \item \(\frac{\partial Q}{\partial x} = \frac{\partial P}{\partial y}\) 在 \(D\) 内恒成立
\end{enumerate}

\subsubsection{全微分求积与全微分方程}

设函数 \(P\),\(Q\) 在单连通区域 \(D\) 上有连续偏导数,且 \(\frac{\partial Q}{\partial x} = \frac{\partial P}{\partial y}\) ,则 \(P\,\mathrm{d}x +Q\,\mathrm{d}y\) 为某函数 \(u\) 的全微分,且取定 \((x_0,y_0)\in D\)

\[
u(x,y) = u(x_0,y_0) + \int_{(x_0,y_0)}^{(x,y)} P\,\mathrm{d}x +Q\,\mathrm{d}y,\quad (x,y)\in D
\]

从而全体函数为 \(u(x,y) + C\)

称求 \(P\,\mathrm{d}x +Q\,\mathrm{d}y\) 的原函数的过程为\textbf{全微分求积}

若 \(P\,\mathrm{d}x +Q\,\mathrm{d}y\) 是某二元函数的全微分,称方程

\[
P(x,y)\,\mathrm{d}x +Q(x,y)\,\mathrm{d}y = 0
\]

为\textbf{全微分方程}

求出一个原函数 \(u(x,y)\),则方程的通解为 \(u(x,y) = C\),其中 \(C\) 是任意常数

\subsection{Gauss 公式和 Stokes 公式}

\subsubsection{Gauss 公式}

\textbf{定理:}

设函数 \(P(x,y,z)\),\(Q(x,y,z)\),\(R(x,y,z)\) 在空间有界闭区域 \(\Omega\) 上有连续偏导数,\(\Omega\) 的边界是光滑或分片光滑的闭曲面 \(\Sigma\),则

\[
\oiint_{\Sigma^+} P\,\mathrm{d}x\mathrm{d}y + Q\,\mathrm{d}y\mathrm{d}z + R\,\mathrm{d}z\mathrm{d}x = \iiint_\Omega \left(\frac{\partial P}{\partial x} + \frac{\partial Q}{\partial y} + \frac{\partial R}{\partial z}\right)\,\mathrm{d}x\mathrm{d}y\mathrm{d}z
\]

- 令 \(P=\frac{x}{3}\),\(Q=\frac{y}{3}\),\(R=\frac{z}{3}\),则可导出 \(V_\Omega = \frac{1}{3}\oiint_{\Sigma^+} x\,\mathrm{d}y\mathrm{d}z + y\,\mathrm{d}z\mathrm{d}x + z\,\mathrm{d}x\mathrm{d}y\),即体积公式
- 使用 Gauss 公式时,注意 \(\Sigma^+\) 的方向应与 \(\Omega\) 的外侧一致

\paragraph{向量形式的 Gauss 公式}

\[
\oiint_{\Sigma^+} \vec{F}\cdot\,\mathrm{d}\vec{S} = \iiint_\Omega \nabla\cdot\vec{F}\,\mathrm{d}x\mathrm{d}y\mathrm{d}z
\]

\subsubsection{通量和散度}

\paragraph{通量}

若给定向量场

\[
\vec{F} = \left(P(x,y,z),Q(x,y,z),R(x,y,z)\right)
\]

则称曲面积分

\[
\Phi = \oiint_{\Sigma^+} \vec{F}\cdot\,\mathrm{d}\vec{S} = \oiint_{\Sigma^+} P\,\mathrm{d}x\mathrm{d}y + Q\,\mathrm{d}y\mathrm{d}z + R\,\mathrm{d}z\mathrm{d}x
\]

为向量场 \(\vec{F}\) 在通过定侧曲面 \(\Sigma^+\) 的\textbf{通量}

\paragraph{散度}

称

\[
\mathrm{div}\vec{F} = \nabla\cdot\vec{F}=\frac{\partial P}{\partial x} + \frac{\partial Q}{\partial y} + \frac{\partial R}{\partial z}
\]

为向量场 \(\vec{F}\) 的\textbf{散度}

则 Gauss 公式可写为

\[
\Phi =\oiint_{\Sigma^+} \vec{F}\cdot\,\mathrm{d}\vec{S} = \iiint_\Omega \mathrm{div}\vec{F}\,\mathrm{d}V
\]

\section{级数}

\subsection{数项级数}

\subsubsection{数项级数的概念}

\textbf{定义:}

给定数列 \(\{a_n\}\) ,和式

\[
\sum_{n = 1}^{\infty} a_n = a_1 + a_2 + \cdots + a_n + \cdots
\]

称为 \textbf{(无穷)级数} ,\(a_n\) 称为级数的\textbf{通项}(或\textbf{一般项})

- \(S_n = \sum_{k = 1}^n a_k\) 称为级数 \(\sum_{n = 1}^{\infty} a_n\) 的前 \(n\) 项\textbf{部分和}
- \(\sum_{k = n + 1}^\infty a_k\) 称为级数 \(\sum_{n = 1}^{\infty} a_n\) 的\textbf{余项级数}

\textbf{定义:}

- 若级数 \(\sum_{n = 1}^\infty a_n\) 的部分和数列 \(\{S_n\}\) 收敛,且 \(\lim_{n\to\infty}S_n = S\),则称级数 \(\sum_{n = 1}^\infty a_n\) \textbf{收敛},\(S\) 称为级数 \(\sum_{n = 1}^\infty a_n\) 的\textbf{和},记作 \(\sum_{n = 1}^\infty a_n = S\)
- 若部分和数列 \(\{S_n\}\) 发散,则称级数 \(\sum_{n = 1}^\infty a_n\) \textbf{发散}

\begin{quote}
常用结论:

\textbf{等比数列} \(\sum_{n = 1}^\infty aq^{n - 1}\begin{cases}\text{收敛于}\frac{a}{1 - q} & ,\vert q \vert < 1\\ \text{发散} & ,\vert q \vert \geq 1\end{cases}\)
\end{quote}

\subsubsection{数项级数的基本性质}

\paragraph{基本性质}

\begin{enumerate}
    \item 若常数 \(\alpha \neq 0\),则级数 \(\sum_{n = 1}^\infty a_n\) 与级数 \(\sum_{n = 1}^\infty \alpha a_n\) 有相同敛散性
    \item \textbf{线性性:} 若级数 \(\sum_{n = 1}^\infty a_n = S\),\(\sum_{n = 1}^\infty b_n = T\),则 \(\forall \alpha, \beta\in \mathbb{R}\),有 \(\sum_{n = 1}^\infty (\alpha a_n + \beta b_n) = \alpha S + \beta T\)
    \item \textbf{可加性:} 将级数增加、删减或改换\textbf{有限项},不改变级数的\textbf{敛散性}
    \item \textbf{结合律:} 若级数收敛于 \(S\),则将相邻若干项添加括号所成新级数仍收敛于 \(S\)
        - 其本质是部分和数列收敛于 \(S\),则子列均收敛于 \(S\)
        - 加括号后级数收敛 \(\nRightarrow\) 原级数收敛
        - 加括号后级数发散 \(\Rightarrow\) 原级数发散
\end{enumerate}

\paragraph{级数收敛的必要条件}

\textbf{定理:}

若 \(\sum_{n = 1}^\infty a_n\) 收敛,则 \(\lim_{n\to\infty} a_n = 0\)

- 若 \(\lim_{n\to\infty} a_n \neq 0 \nRightarrow \sum_{n = 1}^\infty a_n\) 发散
- 若 \(\lim_{n\to\infty} a_n = 0 \nRightarrow \sum_{n = 1}^\infty a_n\) 收敛,比如调和级数

\subsection{正项级数敛散性}

\subsubsection{正项级数}

\textbf{定义:}

若级数 \(\sum_{n = 1}^\infty a_n\) 满足 \(a_n > 0 \quad \left(n\in\mathbb{N}^+\right)\),则称此级数为\textbf{正项级数}

\textbf{定理:(收敛原理)}

正项级数 \(\sum_{n = 1}^\infty a_n\) 收敛 \(\Leftrightarrow\) 是其部分和数列 \(\{S_n\}\) 有上界,即 \(\exists M\in\mathbb{R},\forall n \in \mathbb{N}^+:S_n \leq M\)

\begin{quote}
\(p\) 级数

\(\sum_{n = 1}^\infty \frac{1}{n^p} \begin{cases}\text{收敛} & ,p > 1\\ \text{发散} & ,p \leq 1\end{cases}\)
\end{quote}

\subsubsection{正项级数敛散性判别法}

\paragraph{比较判别法}

\textbf{定理:}

设正项级数 \(\sum_{n = 1}^\infty a_n\),\(\sum_{n = 1}^\infty b_n\) 满足 \(a_n \leq b_n \quad (\forall n \in \mathbb{N}^+)\) ,则 \(\sum_{n = 1}^\infty b_n\) 收敛 \(\Rightarrow \sum_{n = 1}^\infty a_n\) 收敛,\(\sum_{n = 1}^\infty a_n\) 发散 \(\Rightarrow \sum_{n = 1}^\infty b_n\) 发散

- 条件 \(\forall n \in \mathbb{N}^+,a_n \leq b_n\) 可改为 \(\exists N,C > 0  ,\forall n\in \mathbb{N}^+,\forall n \geq N,a_n \leq Cb_n\)
- 使用该判别法时需要有参照级数,常选\textbf{等比级数}或 \textbf{\(p\) 级数}作参照

\paragraph{比较判别法(极限形式)}

\textbf{定理:}

正项级数 \(\sum_{n = 1}^\infty a_n\),\(\sum_{n = 1}^\infty b_n\) 满足 \(\lim_{n\to\infty} \frac{a_n}{b_n} = l\)

- 当 \(0 < l < +\infty\) 时,\(\sum_{n = 1}^\infty a_n\) 与 \(\sum_{n = 1}^\infty b_n\) 同敛散
- 当 \(l = 0\) 时,\(\sum_{n = 1}^\infty b_n\) 收敛 \(\Rightarrow \sum_{n = 1}^\infty a_n\) 收敛
- 当 \(l = +\infty\) 时,\(\sum_{n = 1}^\infty b_n\) 发散 \(\Rightarrow \sum_{n = 1}^\infty a_n\) 发散

\begin{quote}
通常使用 \(b_n = \frac{1}{n^p}\) 作为参照物,因为我们此时在分析无穷小 \(a_n\) 的阶
\end{quote}

\paragraph{比值判别法(d'Alembert 判别法)}

\textbf{定理:}

若\textbf{正项级数} \(\sum_{n = 1}^\infty a_n\) 满足 \(\lim_{n\to\infty} \frac{a_{n + 1}}{a_n} = l\),则

- 当 \(0\leq l < 1\) 时,\(\sum_{n = 1}^\infty a_n\) 收敛
- 当 \(l > 1\) 时,\(\sum_{n = 1}^\infty a_n\) 发散
- 当 \(l = 1\) 时,判别法失效

\begin{quote}
Stirling 公式: \(n! \sim \left(\frac{n}{e}\right)^n \sqrt{2n\pi} \quad (n\to\infty)\)
\end{quote}

\begin{quote}
当 \(a_n\) 是一些乘积构成或含 \(n!\) 时,可以考虑比值法
\end{quote}

\paragraph{根值判别法(Cauchy 判别法)}

\textbf{定理:}

若正项级数 \(\sum_{n = 1}^\infty a_n\) 满足 \(\lim_{n\to\infty} \sqrt[n]{a_n} = l\),则

\begin{enumerate}
    \item 当 \(0\leq l < 1\) 时,\(\sum_{n = 1}^\infty a_n\) 收敛
    \item 当 \(1 < l \leq +\infty\) 时,\(\sum_{n = 1}^\infty a_n\) 发散
    \item 当 \(l = 1\) 时,判别法失效
\end{enumerate}

\begin{quote}
当 \(a_n\) 中含有 \(n\) 次方时,可以考虑使用根值法
\end{quote}

\begin{quote}
比值法和根值法实际上可看作是在将级数与等比级数作比较,均只能判断收敛速度不慢于等比级数的级数。当所求级数存在时,可称级数为\textbf{拟等比级数}
\end{quote}

\begin{quote}
根值法优于比值法

- \(\lim_{n\to\infty}\frac{a_{n + 1}}{a_n} =  l \Rightarrow \lim_{n\to\infty}\sqrt[n]{a_n} = l\)
- \(\lim_{n\to\infty}\sqrt[n]{a_n} = l \nRightarrow \lim_{n\to\infty}\frac{a_{n + 1}}{a_n} =  l\)
\end{quote}

\paragraph{积分判别法}

\textbf{定理:}

设 \(\sum_{n = 1}^\infty a_n\) 为\textbf{正项级数},若非负函数 \(f(x)\) 在 \([1,+\infty)\) 上\textbf{单调递减},且 \(a_n = f(n)\quad \left(\forall n \in \mathbb{N}^+\right)\),则级数 \(\sum_{n = 1}^\infty a_n\) 与反常积分 \(\int_1^{+\infty} f(x)\,\mathrm{d}x\) 有相同的敛散性

- 条件 \(\left[1,+\infty\right)\) 可改为 \(\left[a,+\infty\right),(a > 1)\)

\subsection{任意项级数的敛散性}

\textbf{任意项级数}

正负项分布是任意的级数

\subsubsection{交错级数敛散性的判别法}

\paragraph{交错级数}

\textbf{定义:}

各项正负相间的级数称为\textbf{交错级数},其形式为

\[
\pm\sum_{n = 1}^\infty (-1)^{n - 1}a_n \quad \left(\text{其中}a_n > 0\right)
\]

\paragraph{Leibniz 判别法}

\textbf{定理:}

若交错级数 \(\sum_{n = 1}^\infty (-1)^{n - 1}a_n \,\left(a_n > 0\right)\) 满足:

\begin{enumerate}
    \item \(a_{n + 1} \leq a_n \quad \left(n = 1,2,\ldots\right)\)
    \item \(\lim_{n\to\infty}a_n = 0\)
\end{enumerate}

则级数 \(\sum_{n = 1}^\infty (-1)^{n - 1}a_n\) 收敛,且其余项级数满足

\[
\left\vert \sum_{k = n + 1}^\infty a_k \right\vert \leq a_{n + 1}
\]

\begin{quote}
我们称满足定理条件的级数为 \textbf{Leibniz 型级数}
\end{quote}

\subsubsection{Abel 判别法和 Dirichlet 判别法}

\textbf{定理:(Abel 判别法)}

若 \(\{a_n\}\) 单调且有界,\(\sum_{n = 1}^\infty b_n\) 收敛,则 \(\sum_{n = 1}^\infty a_nb_n\) 收敛

\textbf{定理:(Dirichlet 判别法)}

若 \(\{a_n\}\) 单调趋于 \(0\),\(\sum_{n = 1}^\infty b_n\) 的部分和数列有界,则 \(\sum_{n = 1}^\infty a_nb_n\) 收敛

\subsubsection{绝对收敛与条件收敛}

\textbf{定义:}

设 \(\sum_{n = 1}^\infty a_n\) 为任意项级数

\begin{enumerate}
    \item 若级数 \(\sum_{n = 1}^\infty \vert a_n \vert\) 收敛,则称级数 \(\sum_{n = 1}^\infty a_n\) 为\textbf{绝对收敛}
    \item 若 \(\sum_{n = 1}^\infty \vert a_n \vert\) 发散,而 \(\sum_{n = 1}^\infty a_n\) 收敛,则称 \(\sum_{n = 1}^\infty a_n\) \textbf{条件收敛}
\end{enumerate}

\textbf{定理:}

若 \(\sum_{n = 1}^\infty a_n\) 绝对收敛,则 \(\sum_{n = 1}^\infty a_n\) 收敛

\begin{quote}
\textbf{常用结论:}

\[
\sum_{n = 1}^\infty \frac{(-1)^{n}}{n^p}\begin{cases}\text{绝对收敛} & ,p > 1\\ \text{条件收敛} & , 0 < p \leq 1 \end{cases}
\]
\end{quote}

\textbf{定理:(绝对收敛与条件收敛的本质)}

\begin{enumerate}
    \item 绝对收敛的级数,可以改变任意项的顺序,其收敛性与和均不变(即满足加法交换律)
    \item 条件收敛的级数,总可以适当改变项的顺序,使其按照任意预定的方式收敛或者发散
\end{enumerate}

\subsection{函数项级数}

\textbf{定义:}

设函数列 \(\{u_n(x)\} (n = 1,2,\ldots)\) 在数集 \(X\) 上有定义,则称形式和

\[
\sum_{n = 1}^\infty u_n(x)= u_1(x) + u_2(x) + \cdots + u_n(x) + \cdots 
\]

为\textbf{函数项级数},其中 \(u_n(x)\) 称为\textbf{通项}

\textbf{定义:}

若数项级数 \(\sum_{n = 1}^\infty u_n(x_0)\) 收敛,则 \(x_0\) 为函数项级数 \(\sum_{n = 1}^\infty u_n(x)\) 的一个\textbf{收敛点},否则称为\textbf{发散点},全体收敛点所组成的集合 \(I\) 称为\textbf{收敛域}

\textbf{定义}

记 \(S_n(x) =\sum_{k = 1}^n u_k(x)\) 为 \(\sum_{n = 1}^\infty u_n(x)\) 的前 \(n\) 项\textbf{部分和(函数)},记 \(r_n(x) = \sum_{k = n + 1}^\infty u_k(x)\) 为\textbf{余和}

\textbf{定义:}

对于收敛域 \(I\) 中的任意一点 \(x\),记 \(\sum_{n = 1}^\infty u_n(x)\) 的和为 \(S(x)\),称此函数 \(S(x)\) 为 \(\sum_{n = 1}^\infty u_n(x)\) 的\textbf{和函数}

显然,\(\forall x\in I\),\(\lim_{n\to +\infty}S_n(x) = S(x)\),\(\lim_{n\to +\infty}r_n(x)=0\)

\subsection{幂级数}

\subsubsection{幂级数及其收敛半径}

在函数项级数中,最简单及最重要的级数形如

\[
\sum_{n = 0}^\infty a_n(x - x_0)^n = a_0 + a_1(x - x_0) + a_2(x - x_0)^2 + \ldots + a_n(x - x_0)^n + \ldots
\]

称为\textbf{幂级数},其中常数项 \(a_0,a_1,\ldots,a_n,\ldots\) 称为幂级数的\textbf{系数}

幂级数更一般的形式为 \(\sum_{n = 0}^\infty a_n(x - x_0)^n\)

\paragraph{Abel 定理}

\begin{enumerate}
    \item 若幂级数 \(\sum_{n = 0}^\infty a_n(x - x_0)^n\) 在 \(x = x_0(x\neq 0)\) 收敛,则当 \(\vert x \vert <\vert
\nocite{*}
\bibliography{ref}
\end{document}