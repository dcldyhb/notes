% 数据结构笔记 - Beamer 幻灯片
\documentclass[UTF8, 12pt]{beamer}
\usepackage{ctex} % 支持中文
\usepackage{pgfpages}
\usepackage{graphicx}
\usepackage{xeCJK}
\usepackage{fancyvrb}
\usepackage{color}
\usepackage{longtable,booktabs,array}
\usepackage{caption}
\usepackage{bookmark}
\usepackage{hyperref}
\hypersetup{hidelinks}

% 中文字体设置
\setmainfont{SimSun} % 宋体
\setsansfont{SimHei} % 黑体
\setmonofont{Consolas} % 等宽字体

% 幻灯片主题设置
\usetheme{AnnArbor}
\usecolortheme{seagull}
\setbeamertemplate{navigation symbols}{} % 隐藏导航栏
\setbeamertemplate{caption}[numbered]
\setbeamertemplate{caption label separator}{: }

% 代码块样式定义
\definecolor{codebg}{RGB}{245,245,245}
\newenvironment{codeblock}{\begin{snugshade}\begin{Verbatim}[commandchars=\\\{\}]}{\end{Verbatim}\end{snugshade}}

% 标题与作者信息
\title{数据结构笔记}
\author{[你的姓名]}
\date{\today}

\begin{document}

% 标题页
\begin{frame}
    \titlepage
    \centering
    \href{https://oi-wiki.org/ds}{oiWiki 数据结构参考资料}
\end{frame}

% 内容框架
\begin{frame}{目录}
    \tableofcontents
\end{frame}

% 数据结构概述
\begin{frame}{数据结构概述}
    \begin{block}{定义}
        数据结构是相互之间存在一种或多种特定关系的数据元素的集合,主要研究数据的逻辑结构、存储结构及操作算法。
    \end{block}

    \begin{block}{逻辑结构分类}
        \begin{itemize}
            \item 线性结构:数组、链表、栈、队列
            \item 非线性结构:树、图、哈希表
        \end{itemize}
    \end{block}

    \begin{block}{基本操作}
        \begin{itemize}
            \item 创建、销毁、插入、删除、查找、遍历
            \item 时间复杂度与空间复杂度分析
        \end{itemize}
    \end{block}
\end{frame}

% 线性表示例(顺序表)
\begin{frame}[fragile]{顺序表实现}
    \begin{codeblock}
template <class elemType>
class seqList : public list<elemType> {
private:
    elemType *data;
    int currentLength;
    int maxSize;
    void doubleSpace(); // 扩容函数
public:
    seqList(int initSize = 10);
    ~seqList() { delete[] data; }
    void clear() { currentLength = 0; }
    int length() const { return currentLength; }
    // 其他操作...
};

template <class elemType>
void seqList<elemType>::doubleSpace() {
    elemType *tmp = data;
    maxSize *= 2;
    data = new elemType[maxSize];
    for (int i = 0; i < currentLength; ++i)
        data[i] = tmp[i];
    delete[] tmp;
}
    \end{codeblock}
    \note{顺序表通过连续内存存储元素,支持随机访问,但插入/删除需移动元素,时间复杂度为 O(n)。}
\end{frame}

% 二叉树示例
\begin{frame}[fragile]{二叉树类定义}
    \begin{codeblock}
template <class T>
class binaryTree {
private:
    struct Node {
        T data;
        Node *left, *right;
        Node(T item, Node *L = nullptr, Node *R = nullptr) : data(item), left(L), right(R) {}
    };
    Node *root;

public:
    void preOrder() const { preOrder(root); } // 前序遍历
    void inOrder() const { inOrder(root); }   // 中序遍历
    void postOrder() const { postOrder(root); } // 后序遍历

private:
    void preOrder(Node *t) const;
    void inOrder(Node *t) const;
    void postOrder(Node *t) const;
};
    \end{codeblock}
    \begin{itemize}
        \item 二叉树遍历算法体现递归思想
        \item 可通过队列实现层序遍历
    \end{itemize}
\end{frame}

% 算法分析示例
\begin{frame}{时间复杂度分析}
    \begin{block}{常见复杂度级别}
        \begin{itemize}
            \item O(1):常数时间 - 访问数组元素
            \item O(n):线性时间 - 顺序查找
            \item O(n log n):归并排序、快速排序
            \item O(2^n):指数时间 - 斐波那契数列递归实现
        \end{itemize}
    \end{block}

    \begin{example}{冒泡排序复杂度}
        \begin{codeblock}
void bubbleSort(int a[], int n) {
    for (int i = 0; i < n-1; ++i)
        for (int j = 0; j < n-i-1; ++j)
            if (a[j] > a[j+1]) swap(a[j], a[j+1]);
} // 时间复杂度 O(n²)
        \end{codeblock}
    \end{example}
\end{frame}

% 结尾页
\begin{frame}{总结}
    \begin{itemize}
        \item 数据结构是算法的基础,选择合适结构可优化程序性能
        \item 重点掌握线性表、树、图的逻辑与实现
        \item 多通过 LeetCode 等平台练习算法题
    \end{itemize}
    \vfill
    \centering{感谢观看!}
\end{frame}

\end{document}